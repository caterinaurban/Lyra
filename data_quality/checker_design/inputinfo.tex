\documentclass[11pt]{article}
\usepackage[a4paper, portrait, margin=1in]{geometry}
\usepackage{fancyhdr}
\usepackage{hyperref}
\usepackage{graphicx}
\usepackage{listings}
\usepackage{mathtools}
\usepackage{color}
\usepackage{float}% If comment this, figure moves to Page 2
% Fancy header package for version number
\usepackage[font={small,it}]{caption}

\usepackage{fancyhdr}
%\pagestyle{fancy}
\fancyhf{}
\renewcommand{\headrulewidth}{0pt}
\renewcommand{\footrulewidth}{0pt}
\newcommand{\bigcell}[2]{\begin{tabular}{@{}#1@{}}#2\end{tabular}}
 
\lstdefinelanguage{Python2}{
  language     = Python,
  morekeywords = {assert},
}
 
\lstdefinestyle{mystyle}{
    backgroundcolor=\color{white},   
    commentstyle=\color{black},
    keywordstyle=\color{blue},
    numberstyle=\tiny\color{black},
    stringstyle=\color{black},
    basicstyle=\footnotesize,
    breakatwhitespace=false,         
    breaklines=true,                 
    captionpos=b,                    
    keepspaces=true,                 
    numbers=left,                    
    numbersep=5pt,                  
    showspaces=false,                
    showstringspaces=false,
    showtabs=false,                  
    tabsize=2
}

\lstset{style=mystyle, language=Python2}

\begin{document}

\pagenumbering{arabic}

\section{Inputfile access}

\begin{itemize}
\item \textit{inputfile = open(filename)} 

Input is accessed with open(). 

Examples: homeworks/dnaAnalyzer, homeworks/mcb112/sandmice\_Harleen

\item \textit{inputfile = open(filename, 'r')} 

Input is accessed with open(). 

Examples: Y16Round1A/alkjash10, homeworks/1516b/ex5\_soln, homeworks/1516b/ex10\_soln

\item \textit{T = input()} 

Input is accessed with input(). 

Examples: Quali15/Cypi00, Quali16/rowdark20, Quali16/cgy4ever20, Quali16/claram00, Quali16/Endagorian20, Quali16/h4tguy20, Quali16/rowdark20, Quali17/JasperH00, Quali17/Nore00, Quali17/Nore20, Y17Round1A/Nore10

\item \textit{with open(filename) as inputfile:} 

Input is accessed with open(). 

Examples: homeworks/cis192/tadas412
\end{itemize}

\section{Input read}

\begin{itemize}
\item \textit{T = input()} 

\textbf{\textasciicircum.*\$} 

Line is read, can be empty

Waiting for input if using command line
EOFError when using "cat inputfile | python readinput.py" and inputfile is empty

Examples: Quali16/cgy4ever20, Quali16/claramm00, Quali16/Endagorian20

\item \textit{T = int(input())} 

\textbf{\textasciicircum\textbackslash{}s*((-$|$\textbackslash{}+)?\textbackslash{}d+)\textbackslash{}s*\$} 

Line with one integer with possible whitespaces at beginning or end.

ValueError if different than regex

Examples: Quali15/Cypi00, Quali16/rowdark20, Quali16/claramm00, Quali16/h4tguy20, Quali17/JasperH00, Quali17/Nore00, Quali17/Nore20, Y17Round1A/Nore10, Y17Round1B/JongMan00

\item \textit{T = int(inputfile.readline())} 

\textbf{\textasciicircum\textbackslash{}s*((-$|$\textbackslash{}+)?\textbackslash{}d+)\textbackslash{}s*\$} 

Line with one integer with possible whitespaces at beginning or end.

ValueError if different than regex

Examples: Y16Round1A/alkjash10, homeworks/1516b/ex10\_soln

\item \textit{x = inputfile.readlines()} 

All lines of a file stored in an array

No error

Examples: homeworks/1516b/ex10\_soln

\item \textit{x, y = input().split()}

\textbf{\textasciicircum\textbackslash{}s*(\textbackslash{}S+)\textbackslash{}s+(\textbackslash{}S+)\textbackslash{}s*\$} 

Line with two inputs separated by whitespaces with possible whitespaces at beginning or end.

ValueError if not like regex

Examples: Quali15/Cypi00

\item \textit{x, y = map(int, input().strip().split())}

strip() (or rstrip()) not necessary, makes no difference since split() ignores all whitespaces

\textbf{\textasciicircum\textbackslash{}s*((-$|$\textbackslash{}+)?\textbackslash{}d+)\textbackslash{}s+((-$|$\textbackslash{}+)?\textbackslash{}d+)\textbackslash{}s*\$} 

Line with two integers separated by whitespaces with possible whitespaces at beginning or end.

ValueError if not like regex

Examples: Quali16/rowdark20, Quali16/Endagorian20, Quali16/h4tguy20, Quali16/cgy4ever20, Quali17/JasperH00 Quali17/Nore00, Quali17/Nore20, Y17Round1A/Nore10, Y17Round1B/JongMan00

\item \textit{x = map(int, inputfile.readline().split())}

\textbf{\textasciicircum(\textbackslash{}s*((-$|$\textbackslash{}+)?\textbackslash{}d*)\textbackslash{}s*)*\$}

or \textbf{\textasciicircum\textbackslash{}s*(?P$<$A$>$(-$|$\textbackslash{}+)?\textbackslash{}d+)\textbackslash{}s+(?P$<$B$>$(-$|$\textbackslash{}+)?\textbackslash{}d+)\textbackslash{}s+(?P$<$C$>$(-$|$\textbackslash{}+)?\textbackslash{}d+) ... \textbackslash{}s+(?P$<$Z$>$(-$|$\textbackslash{}+)?\textbackslash{}d+)\$}

Line with any number of integers separated by whitespaces with possible whitespaces at beginning or end.

ValueError if not like regex

Examples: Y16Round1A/alkjash10

\item \textit{x = line.split(',')}

\textbf{\textasciicircum.*(,.*)*\$} 

or \textbf{\textasciicircum(?P$<$A$>$[\textasciicircum,]*)(,(?P$<$B$>$[\textasciicircum,]*))(,(?P$<$C$>$[\textasciicircum,]*)) ... (,(?P$<$Z$>$[\textasciicircum,]*))\$}

Line with inputs separated by comma

No error

Examples: homeworks/1516b/ex5\_soln

\item \textit{import csv; x = [v for v in csv.reader(inputfile, delimiter='\textbackslash{}t')]}

\textbf{\textasciicircum.*(\textbackslash{}t.*)*\$} 

or \textbf{\textasciicircum(?P$<$A$>$[\textasciicircum\textbackslash{}t]*)(\textbackslash{}t(?P$<$B$>$[\textasciicircum\textbackslash{}t]*))(\textbackslash{}t(?P$<$C$>$[\textasciicircum\textbackslash{}t]*)) ... (\textbackslash{}t(?P$<$Z$>$[\textasciicircum\textbackslash{}t]*))\$}

Line with inputs separated by \textbackslash{}t

No error

Examples: homeworks/dat8/homework3\_chipotle

\item \textit{for n in range(N): ...input()...}

N lines with same structure

Nothin read if N <= 0

Examples: Quali15/Cypi00, Quali16/claramm00, Quali16/rowdark20, Quali17/JasperH00, Quali17/Nore00, Quali17/Nore20, Y16Round1A/alkjash10, Y17Round1A/Nore10, Y17Round1B/JongMan00

\item \textit{for line in inputfile:}

N lines with same structure

No Error

Examples: homeworks/1516b/ex5\_soln

\item \textit{fields = line.strip().split(","); x = fields[1]}

\textbf{\textasciicircum.*,.*\$} 

Each line in file must have at least one comma

IndexError if not like regex

Examples: homeworks/1516b/ex10\_soln

\end{itemize}

\begin{minipage}{\linewidth}
  \centering
\begin{lstlisting}[language=Python2]
tsv = open(args.input, "r")
line_num = 0
info_keys = []
infos = []
for line in tsv:
    line_num += 1
    if line_num == 1:
	# first line stripped must contain 'Genomic_Coordinate'
	info_keys = line.strip().split(args.delimiter)
    else:
	info_values = line.strip().split(args.delimiter)
	# info_keys must contain 'Genomic_Coordinate'
	# for element info_values[info_keys.index('Genomic_Coordinate)]:
	# 	len(element.split(":")[0]) > 3
	# 	len(element.split(":") > 2) ==> info["Genomic_Coordinate"] must contain at least one ':'
	# 	len(element.split(":")[2].split(">")) > 1 ==> info["Genomic_Coordinate"].split(":")[2] must contain at least one '>' ==> info["Genomic_Coordinate"] must contain at least two ':'
	# 	element.split(":")[0][3:] must be int
	# 	element.split(":")[1] must be int
	infos.append(dict(zip(info_keys, info_values)))
# FORALL info in infos:
# 	info["Genomic_Coordinate"].split(":")[0][3:] must be '13' or '17'
info_dict = {13:{}, 17:{}}
# FORALL info in infos:
# 	info must contain 'Genomic_Coordinate'
# 	len(info["Genomic_Coordinate"].split(":")[0]) > 3
# 	len(info["Genomic_Coordinate"].split(":") > 2) ==> info["Genomic_Coordinate"] must contain at least one ':'
# 	len(info["Genomic_Coordinate"].split(":")[2].split(">")) > 1 ==> info["Genomic_Coordinate"].split(":")[2] must contain at least one '>' ==> info["Genomic_Coordinate"] must contain at least two ':'
# 	info["Genomic_Coordinate"].split(":")[0][3:] must be int
# 	info["Genomic_Coordinate"].split(":")[1] must be int
# info_dict must contain info["Genomic_Coordinate"].split(":")[0][3:]
for info in infos:
    # info must contain 'Genomic_Coordinate'
    # len(info["Genomic_Coordinate"].split(":")[0]) > 3
    # len(info["Genomic_Coordinate"].split(":") > 2) ==> info["Genomic_Coordinate"] must contain at least one ':'
    # len(info["Genomic_Coordinate"].split(":")[2].split(">")) > 1 ==> info["Genomic_Coordinate"].split(":")[2] must contain at least one '>' ==> info["Genomic_Coordinate"] must contain at least two ':'
    # info["Genomic_Coordinate"].split(":")[0][3:] must be int
    # info_dict must contain info["Genomic_Coordinate"].split(":")[0][3:]
    # info["Genomic_Coordinate"].split(":")[1] must be int
    items = info["Genomic_Coordinate"].split(":")
    # len(items[0]) > 3
    # items[0][3:] must be int
    # info_dict must contain items[0][3:]
    chrom = items[0][3:]
    # items[1] must be int
    pos = items[1]
    ref = items[2].split(">")[0]
    # len(items) > 2
    # len(items[2].split(">")) > 1 ==> items[2] must contain at least one '>'
    alt = items[2].split(">")[1]
    # chrom must be int
    chrom = int(chrom)
    # pos must be int
    pos = int(pos)
    # info_dict must contain chrom
    if pos not in info_dict[chrom]:
	info_dict[chrom][pos] = [info]
    else:
	info_dict[chrom][pos].append(info)
\end{lstlisting}
\end{minipage}

\end{document}

















