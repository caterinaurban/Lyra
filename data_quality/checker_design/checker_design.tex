\documentclass[11pt]{article}
\usepackage[a4paper, portrait, margin=1in]{geometry}
\usepackage{fancyhdr}
\usepackage{hyperref}
\usepackage{graphicx}
\usepackage{listings}
\usepackage{mathtools}
\usepackage{listings}
\usepackage{color}
\usepackage[dvipsnames]{xcolor}
\usepackage{float}% If comment this, figure moves to Page 2
% Fancy header package for version number
\usepackage[font={small,it}]{caption}

\usepackage{fancyhdr}

%\pagestyle{fancy}
\fancyhf{}
\renewcommand{\headrulewidth}{0pt}
\renewcommand{\footrulewidth}{0pt}
\newcommand{\bigcell}[2]{\begin{tabular}{@{}#1@{}}#2\end{tabular}}
 
\lstdefinelanguage{Python2}{
  language     = Python,
  morekeywords = {assert},
}
 
\lstdefinestyle{mystyle}{
    backgroundcolor=\color{white},   
    commentstyle=\color{ForestGreen},
    keywordstyle=\color{blue},
    numberstyle=\tiny\color{black},
    stringstyle=\color{red},
    basicstyle=\footnotesize,
    breakatwhitespace=false,         
    breaklines=true,                 
    captionpos=b,                    
    keepspaces=true,                 
    numbers=left,                    
    numbersep=5pt,                  
    showspaces=false,                
    showstringspaces=false,
    showtabs=false,                  
    tabsize=2
}

\lstset{style=mystyle, language=Python2}

\begin{document}

\title{Checker Design}
\maketitle

\pagenumbering{arabic}

\section{Error Correction}

User needs some means to correct the error. 

\begin{itemize}
  \item After the user did the corrections in the program we write the changes to the input file and rerun the checker.
  \item Idea: We keep a separate file where the input data is copied to and add placeholders for input data that needs correction. This way when scanning the file again to write down the corrected values we know where we have to check for new values.
  
  Problem: What if the user makes additional changes in the original file. A developer also corrects his own code when error messages appear and copies it if necessary. It is also easy to show how to do it with a separate file. We would just have to tell the user that only changes in the new file are taken into account. But if the user can change the file (delete a row etc) we cannot add placeholders because it would be confusing. So we just order the new values internally by location. If the user changes anything in the file manually the changes will only be seen if the checker is rerun.
\begin{itemize}


\section{Steps}

\begin{enumerate}
\item JSON Input: \verb|{"0": {"0": {"strict":"True", "assmp":[Any, [-inf, inf]]}, |

\verb|"10":{"strict":"True", "assmp":[Int, [0, inf]]}, |

\verb|"11":{"strict":"True", "assmp":[Any, [-inf, inf]]}, |

\verb|"len_a":{"strict":"False", "assmp":[]}, "delimiter":";"}, |

\verb|"len_b": {"strict":"False"}, "filename":"/home/user/documents/input.txt"}|
\item read input from JSON into dict
\item preprocess data to be better accessable (maybe as (simplified) lattice elements)
\item create additional elements for simplified accessing of correct line (for example create [0,10,11] to know when input information changes, or checking the number of elements a line should at least have (by finding the highest bound where "strict":"True"))
\item get information for first line
\item iterate through lines of file with given filename (and iteration variable i)
\begin{enumerate}
\item get information for current line (same as last line as long as i is smaller than next higher variable in dictionary)
\item split current line by given delimiter
\item check if the number of elements is equal or higher to the minimum number stated by the assumption
\item IF NOT satisfied: create an error message and check next line
\item get information about first element
\item iterate through elements (with iteration variable j)
\begin{enumerate}
\item get information for current element (same as last element as long as j is smaller than next higher variable in dictionary)
\item check current assumptions for current element
\item IF satisfied: go on with next element
\item IF NOT satisfied: create error message
\end{enumerate}
\end{enumerate}
\end{enumerate}

\section{Design}

\begin{itemize}
\item \textit{Assumption Domain:}
\begin{itemize}
\item does: analysis
\item result: final assumption of input as INPUTS = ListAbstraction()
\end{itemize}

\item \textit{JSONHandler:} 
\begin{itemize}
\item does: writes assumptions to JSON, reads JSON into format understandable by input checker
\item result: creates JSON file, returns assumptions for input checker
\end{itemize}


\item \textit{Input Checker:} 
\begin{itemize}
\item does: uses assumptions prepared by JSONHandler to check the input for violations, reports errors if necessary
\item result: error messages or "all is fine"
\end{itemize}


\item \textit{Runner:} 
\begin{itemize}
\item does: runs the analysis and input checker
\end{itemize}

\end{itemize}

\end{document}

















