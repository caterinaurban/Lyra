\documentclass[11pt]{article}
\usepackage[a4paper, portrait, margin=1in]{geometry}
\usepackage{fancyhdr}
\usepackage{hyperref}
\usepackage{graphicx}
\usepackage{listings}
\usepackage{mathtools}
\usepackage{color}
\usepackage[dvipsnames]{xcolor}
\usepackage{float}% If comment this, figure moves to Page 2
% Fancy header package for version number
\usepackage[font={small,it}]{caption}

\usepackage{fancyhdr}
%\pagestyle{fancy}
\fancyhf{}
\renewcommand{\headrulewidth}{0pt}
\renewcommand{\footrulewidth}{0pt}
\newcommand{\bigcell}[2]{\begin{tabular}{@{}#1@{}}#2\end{tabular}}
 
\lstdefinelanguage{Python2}{
  language     = Python,
  morekeywords = {assert},
}
 
\lstdefinestyle{mystyle}{
    backgroundcolor=\color{white},   
    commentstyle=\color{ForestGreen},
    keywordstyle=\color{blue},
    numberstyle=\tiny\color{black},
    stringstyle=\color{red},
    basicstyle=\footnotesize,
    breakatwhitespace=false,         
    breaklines=true,                 
    captionpos=b,                    
    keepspaces=true,                 
    numbers=left,                    
    numbersep=5pt,                  
    showspaces=false,                
    showstringspaces=false,
    showtabs=false,                  
    tabsize=2
}

\lstset{style=mystyle, language=Python2}

\begin{document}

\title{Example to Show}

\maketitle

\pagenumbering{arabic}

\section{Problem to solve}

\begin{itemize}

\item \textit{for i in range(N): y = a[i]} At what point should information about i ([0,N-1]) be incorporated into analysis?


\item maybe it's better to have \verb|LINES = {0} ({0} int {1} int {2}) {len_LINES}| and if later there is another \verb|input()| we just add it in front: 

\verb|LINES = {0} T {1} ({0} int {1} int {2}) {len_LINES}|

instead of having \verb|LINES = {0} T {len_LINES-N} ({0} int {1} int {2}) {len_LINES}| 

and then \verb|LINES = {0} T {len_LINES-N-1} T {len_LINES-N} ({0} int {1} int {2}) {len_LINES}| 

and then \verb|LINES = {0} T {len_LINES-N} ({0} int {1} int {2}) {len_LINES}|


\item How to substitute arrays using i when the $i \in [x,y]$ is known:

\verb|a = {0} T {i} int {len_a}| should result in \verb|a = {0} T {x} int {len_a}|

\verb|a = {0} T {i} int {i+1} T {len_a}| should result in \verb|a = {0} T {x} int {y+1} T {len_a}|

\verb|a = {0} int {i+1} T {len_a}| should result in \verb|a = {0} int {y+1} T {len_a}|

\verb|a = {0} T {i} int {i+1} T {i+4} int {i+5} T {len_a}| should result in 

\verb|a = {0} T {x} int {x+1} int {x+2} int {x+3} int {x+4} int {x+5} int {x+6} int| 
\verb|{x+7} int {x+8} int ... int {y+5} T {len_a}|

simplest solution: just add for every element in [x,y]. more advanced: make case destinctions. Problem: If [x,y] are variables we dont know how many. simplest solution: just lose information in that case.


\item \textit{for i in range(): y == int(a[i])} In first iteration we get \verb|a = {0} T {i}? int {i+1} T {len}?|. In second iteration do we we get \verb|a = {0} T {i}? int {i+1}? int {i+2} T {len}?| or \verb|a = {0} T {i}? int {i+1} int {i+2} T {len}?|. Because in the second loop we know that if we access element i the element i+1 must be strictly greater and exist. Or do we keep the ? on the element?

\end{itemize}

\section{Patterns to include in example}

\begin{itemize}

\item for line in data:

\item splitting line by delimiter

\item casting values to int with map(int, splittedelements) or n = int()

\item accessing values in array by hardcoded number (most common is 0), depending on other line (because all lines should be same size as first line)

\item slicing because of [0] is label and [1:] is data

\item doing some analysis on values based on what condition they fullfil (example: if line starts with '$>$' do that)

\item iterating using range() with size of first vector

\end{itemize}


\section{Example3}

\lstset{escapeinside={(*@}{@*)}}
\begin{lstlisting}
data = [] (*@\label{Example3step12}@*) 
for line in open('input3.txt', 'r'): (*@\label{Example3step11}@*) 
    data.append(line.split(",")) (*@\label{Example3step10}@*) 

grades = []

for fields in data: (*@\label{Example3step9}@*) 
    if fields[1] == 'midterms': (*@\label{Example3step8}@*) 
        grade = (int(fields[2]) + 2 * int(fields[3])) / 3 (*@\label{Example3step7}@*) 
        grades.append(grade) (*@\label{Example3step6}@*) 
    else:
        grades.append(int(fields[2])) (*@\label{Example3step5}@*) 

for grade in grades: (*@\label{Example3step4}@*) 
    if grade < 1 or grade > 6: (*@\label{Example3step3}@*) 
        raise ValueError('Invalid grade!') (*@\label{Example3step2}@*) 

print('class average:', sum(grades) / len(grades)) (*@\label{Example3step1}@*) 
 (*@\label{Example3step0}@*) 
\end{lstlisting}



\section{Automation Example3}

\begin{itemize}

\item \textit{line \ref{Example3step0}:} Array are initialized with 

\verb|grades = T| and \verb|fields = T| and \verb|data = T|

\item \textit{line \ref{Example3step1}:} len(grades) $>$ 0 $==>$ grades cannot be empty:

\verb|grades = {0} T {len_grades}|

\item \textit{line \ref{Example3step2}:} Bad state: BOTTOM

\item \textit{line \ref{Example3step3}:} In order not to reach bad state need to negate condition: $grade >= 1$ and $grade <= 6$

\verb|grade == int and grade == [1,6]|

\item \textit{line \ref{Example3step4}:} for-loop header reached, iteration limit not reached, no fixed point, no loop variable

\item \textit{line \ref{Example3step2} (2nd round):} Bad state: BOTTOM

\item \textit{line \ref{Example3step3} (2nd round):} In order not to reach bad state need to negate condition: $grade >= 1$ and $grade <= 6$

\verb|grade == int and grade == [1,6]|

\item \textit{line \ref{Example3step4} (2nd round):} for-loop header reached, fixed point reached. condition for grade has to hold for every element in grades:

\verb|grades = {0} int, [1,6] {len_grades}|

\item \textit{line \ref{Example3step5}:} array access: \verb|fields = {0} T {2} T {3} T {len_fields}?|

int cast: \verb|fields = {0} T {2} int {3} T {len_fields}?|

element of grades: \verb|fields = {0} T {2} int, [1,6] {3} T {len_fields}?|

\item \textit{line \ref{Example3step6}:} element of grades: \verb|grade == int and grade == [1,6]|

\item \textit{line \ref{Example3step7}:} array access at 2: \verb|fields = {0} T {2} T {3} T {len_fields}?|

int cast: \verb|fields = {0} T {2} int {3} T {len_fields}?|

array access at 3: \verb|fields = {0} T {2} int {3} T {4} T {len_fields}?|

int cast: \verb|fields = {0} T {2} int {3} int {4} T {len_fields}?|

more advanced: grade has to be in [1,6]: 

\verb|fields[2] = [1,6] * 3 - 2 * fields[3] = [3,18] - 2 * fields[3]| and

\verb|fields[3] = ([1,6] * 3 - fields[2]) / 2 = ([3,18] - fields[2]) / 2|

\item \textit{line \ref{Example3step8}:} Merging if and else branch: 

\verb|fields = {0} T {2} int, [1,6] {3} T {len_fields}?| and 

\verb|fields = {0} T {2} int {3} int {4} T {len_fields}?| results in

\verb|fields = {0} T {2} int {3} T {len_fields}?|

Underapproximation: \verb|fields = {0} T {2} int, [1,6] {6} T {len_fields}?|

Exact: \verb|IF fields[1] == 'midterms' THEN fields = {0} T {2} int, [1,6] {6} T {len_fields}?|

\verb|ELSE fields = {0} T {2} int {3} int {4} T {len_fields}?|

array access in condition: \verb|fields = {0} T {1} T {2} int {3} T {len_fields}?|

\item \textit{line \ref{Example3step9}:} for-loop header reached, iteration limit not reached, no fixed point, no loop variable

...

\item \textit{line \ref{Example3step9}:} for-loop header reached, fixed point reached. condition for line has to hold for every element in data:

\verb|data = {0} ({0} T {1} T {2} int {3} T {len_fields}?) {len_data}?|


\item \textit{line \ref{Example3step11}:} Getting information about input: 

\verb|line = {0} T {len_line}? with delimiter ','|

element of data: \verb|line = {0} T {1} T {2} int {3} T {len_fields}? with delimiter ','|

\item \textit{line \ref{Example3step11}:} input is from file 'input3.txt': \verb|FILE: input.txt|

line holds for every element in input: 

\verb|LINES = {0} ({0} T {1} T {2} int {3} T {len_fields}? with delimiter ',') {len_LINES}?|

\item \textit{final json:} \verb|{"filename":"input2.txt",|

\verb|"0":{"0":"[]", "1":"[]", "2":["int"], "3":"[]", "?":"", "del":","} "?":""}|



\end{itemize}


\section{Example2}


\lstset{escapeinside={(*@}{@*)}}
\begin{lstlisting}
f = open('input2.txt', 'r') (*@\label{Example2step13}@*) 
first_line = f.readline() (*@\label{Example2step12}@*) 
lines = f.readlines() (*@\label{Example2step11}@*) 
f.close()

grades = []

for line in lines: (*@\label{Example2step10}@*) 
    fields = line.split(",") (*@\label{Example2step9}@*) 
(*@\label{Example2step8}@*) 
    if fields[1] == 'hw':
        for grade in fields[2:6]: (*@\label{Example2step7}@*) 
            grades.append(int(grade)) (*@\label{Example2step6}@*) 
    else:
        grades.append(int(fields[2])) (*@\label{Example2step5}@*) 

for grade in grades: (*@\label{Example2step4}@*) 
    if grade < 1 or grade > 6: (*@\label{Example2step3}@*) 
        raise ValueError('Invalid grade!') (*@\label{Example2step2}@*) 

print('class average:', sum(grades) / len(grades)) (*@\label{Example2step1}@*) 
(*@\label{Example2step0}@*) 
\end{lstlisting}

\section{Automation Example2}

\begin{itemize}

\item \textit{line \ref{Example2step0}:} Array are initialized with 

\verb|grades = T| and \verb|fields = T|

\item \textit{line \ref{Example2step1}:} len(grades) $>$ 0 $==>$ grades cannot be empty:

\verb|grades = {0} T {len_grades}|

\item \textit{line \ref{Example2step2}:} Bad state: BOTTOM

\item \textit{line \ref{Example2step3}:} In order not to reach bad state need to negate condition: $grade >= 1$ and $grade <= 6$

\verb|grade == [1,6]|

\item \textit{line \ref{Example2step4}:} for-loop header reached, iteration limit not reached, no fixed point, no loop variable

\item \textit{line \ref{Example2step2} (2nd round):} Bad state: BOTTOM

\item \textit{line \ref{Example2step3} (2nd round):} In order not to reach bad state need to negate condition: $grade >= 1$ and $grade <= 6$

\verb|grade == [1,6]|

\item \textit{line \ref{Example2step4} (2nd round):} for-loop header reached, fixed point reached. condition for grade has to hold for every element in grades:

\verb|grades = {0} [1,6] {len_grades}|

\item \textit{line \ref{Example2step5}:} array access: \verb|fields = {0} T {2} T {3} T {len_fields}?|

int cast: \verb|fields = {0} T {2} int {3} T {len_fields}?|

element of grades: \verb|fields = {0} T {2} int, [1,6] {3} T {len_fields}?|

\item \textit{line \ref{Example2step6}:} \verb|grade == int|

element of grades: \verb|grade == int AND grade == [1,6]|

\item \textit{line \ref{Example2step7}:} for-loop header reached, iteration limit not reached, no fixed point, no loop variable

\item \textit{line \ref{Example2step6}:} \verb|grade == int|

element of grades: \verb|grade == int AND grade == [1,6]|

\item \textit{line \ref{Example2step7}:} for-loop header reached, fixed point reached. condition for grade has to hold for $fields[2:6]$:

\verb|fields = {0} T {2} int, [1,6] {6} T {len_fields}?|

\item \textit{line \ref{Example2step8}:} Merging if and else branch: 

\verb|fields = {0} T {2} int, [1,6] {3} T {len_fields}?| and 

\verb|fields = {0} T {2} int, [1,6] {6} T {len_fields}?| results in

\verb|fields = {0} T {2} int, [1,6] {3} T {len_fields}?|

Underapproximation: \verb|fields = {0} T {2} int, [1,6] {6} T {len_fields}?|

Exact: \verb|IF fields[1] == 'hw' THEN fields = {0} T {2} int, [1,6] {6} T {len_fields}?|

\verb|ELSE fields = {0} T {2} int, [1,6] {3} T {len_fields}?|

array access: \verb|fields = {0} T {1} T {2} int, [1,6] {3} T {len_fields}?|

\item \textit{line \ref{Example2step9}:} Getting information about input: 

\verb|line = {0} T {1} T {2} int, [1,6] {3} T {len_fields}? with delimiter ','|

\item \textit{line \ref{Example2step10}:} for-loop header reached, iteration limit not reached, no fixed point, no loop variable

...

\item \textit{line \ref{Example2step10}:} for-loop header reached, fixed point reached. condition for line has to hold for every elemen in lines:

\verb|lines = {0} ({0} T {1} T {2} int, [1,6] {3} T {len_fields}?) {len_lines}? with delimiter ','|

\item \textit{line \ref{Example2step11}:} input is read as lines from file f:

\verb|LINES = {0} ({0} T {1} T {2} int, [1,6] {3} T {len_fields}? with delimiter ',') {len_lines}?|

\verb|FILE: f|

\item \textit{line \ref{Example2step12}:} another input is read from file f:

\verb|LINES = {0} T {1} ({0} T {1} T {2} int, [1,6] {3} T {len_fields}? with delimiter ',') {len_lines}?|

\verb|FILE: f|

\item \textit{line \ref{Example2step13}:} filename is input2.txt:

\verb|FILE: input2.txt|

\item \textit{final json:} \verb|{"filename":"input2.txt",|

\verb|"0":"T", "1":{"0":"T", "2":["int", ">= 1", "<= 6"], "3":"T", "?":"", "del":","}}|



\end{itemize}
% 
\section{Example}

\lstset{escapeinside={(*@}{@*)}}
\begin{lstlisting}
f = open('input.txt', 'r')
first_line = f.readline()
lines = f.readlines()
f.close()

names = []
grades = []

for line in lines:
    fields = line.split(",")
    names.append(fields[0])
    curr_grades = []
    if len(fields) == 5:
        for grade in fields[1:5]:
            curr_grades.append(int(grade))
    else:
        for i in range(4):
            curr_average = (int(fields[i+1]) + int(fields[i+5])) / 2
            curr_grades.append(curr_average)
    grades.append(curr_grades) (*@\label{step5}@*) 

print()
print('averages:')
(*@\label{step4}@*) 
for i in range(len(names)): (*@\label{step3}@*) 
    average_grade = sum(grades[i]) / len(grades[i]) (*@\label{step2}@*) 
    print(names[i], average_grade) (*@\label{step1}@*) 
(*@\label{step0}@*) 
\end{lstlisting}

\section{Automation Example}

\begin{itemize}

\item \textit{line \ref{step0}} Array are initialized with 

\verb|names = T| and \verb|grades = T| and \verb|curr_grades = T| and \verb|fields = T|

\item \textit{line \ref{step1}} There is an array access so 

\verb|names = {0} T {i}? T {i+1} T {len_names}?|

\item \textit{line \ref{step2}} There is an array access so 

\verb|grades = {0} T {i}? T {i+1} T {len_grades}?|. Furthermore sum() is used for array access, so 

\verb|grades = {0} T {i}? int {i+1} T {len_grades}?|. Furthermore len() is used for array access, so 

\verb|grades = {0} T {i}? int {i+1} T {len_grades}?|. 

\item \textit{line \ref{step3}} for-loop with range() is encountered, so iteration variable i is now i+1: 

\verb|names = {0} T {i+1}? T {i+2} T {len_names}?| and 

\verb|grades = {0} T {i+1}? int {i+2} T {len_grades}?|

\item \textit{line \ref{step1} (2nd time)} There is an array access so 

\verb|names = {0} T {i}? T {i+1} T {i+2} T {len_names}?|. Joining this with last round we get 

\verb|names = {0} T {i}? T {i+1} T {len_names}?|

\item \textit{line \ref{step2} (2nd time)} There is an array access so 

\verb|grades = {0} T {i} T {i+1} T {i+2} T {len_grades}?|. 

Furthermore sum() is used for array access, so 

\verb|grades = {0} T {i} int {i+1} int {i+2} T {len_grades}?|. 

Furthermore len() is used for array access, so 

\verb|grades = {0} T {i}? int {i+1} int {i+2} T {len_grades}?|.

Joining this with last round we get \verb|grades = {0} T {i}? int {i+1} T {len_grades}?|.

\item \textit{line \ref{step3} (2nd time)} for-loop with range() is encountered, we did not reach a fix point or the limit of loop iterations, so iteration variable i is now i+1: 

\verb|names = {0} T {i+1}? T {i+2} T {len_names}?| and 

\verb|grades = {0} T {i+1}? int {i+2} T {len_grades}?|

\item \textit{line \ref{step1} (3rd time)} There is an array access so 

\verb|names = {0} T {i}? T {i+1} T {i+2} T {len_names}?|. 

Joining this with last round we get 

\verb|names = {0} T {i}? T {i+1} T {len_names}?|

\item \textit{line \ref{step2} (3rd time)} There is an array access so 

\verb|grades = {0} T {i}? T {i+1} T {i+2} T {len_grades}?|. 

Furthermore sum() is used for array access, so 

\verb|grades = {0} T {i}? int {i+1} int {i+2} T {len_grades}?|. 

Furthermore len() is used for array access, so 

\verb|grades = {0} T {i}? int {i+1} int {i+2} T {len_grades}?|.  

Joining this with last round we get \verb|grades = {0} T {i}? int {i+1} T {len_grades}?|.

\item \textit{line \ref{step3} (3nd time)} for-loop with range() is encountered, we reached a fix point. We have know from range that $i \in [0, len(names)-1]$. Since names is an array we know that the len(names) $==$ len\_names. So $i \in [0, len\_names-1]$. We therefore check all our arrays to substitute i with [0, len\_names-1]: 

\verb|names = {0} T {0}? T {len_names} T {len_names}?| and 

\verb|grades = {0} T {0}? int {len_names} T {len_grades}?|

\item \textit{line \ref{step4}} Loop simplification:

\verb|names = {0} T {len_names} T {len_names}?| and 

\verb|grades = {0} int {len_names} T {len_grades}?|

Merging loop with else branch:
\verb|grades = T| and \verb|grades = {0} int {len_names} T {len_grades}?| results in \verb|grades = {0} int {len_names}? T {len_grades}?|

\verb|names = T| and \verb|names = {0} T {len_names} T {len_names}?| results in \verb|names = {0} T {len_names}?|

\item \textit{line \ref{step5}} 

\verb|grades = {0} int {len_names}? T {len_grades}?|

\end{itemize}
\end{document}





















