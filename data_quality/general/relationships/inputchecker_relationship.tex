\documentclass[11pt]{article}
\usepackage[a4paper, portrait, margin=1in]{geometry}
\usepackage{fancyhdr}
\usepackage{hyperref}
\usepackage{graphicx}
\usepackage{listings}
\usepackage{mathtools}
\usepackage{color}
\usepackage[dvipsnames]{xcolor}
\usepackage{float}
\usepackage[font={small,it}]{caption}

\usepackage{fancyhdr}
%\pagestyle{fancy}
\fancyhf{}
\renewcommand{\headrulewidth}{0pt}
\renewcommand{\footrulewidth}{0pt}
\newcommand{\bigcell}[2]{\begin{tabular}{@{}#1@{}}#2\end{tabular}}
 
\lstdefinelanguage{Python2}{
  language     = Python,
  morekeywords = {assert},
}
 
\lstdefinestyle{mystyle}{
    backgroundcolor=\color{white},   
    commentstyle=\color{ForestGreen},
    keywordstyle=\color{blue},
    numberstyle=\tiny\color{black},
    stringstyle=\color{red},
    basicstyle=\footnotesize,
    breakatwhitespace=false,         
    breaklines=true,                 
    captionpos=b,                    
    keepspaces=true,                 
    numbers=left,                    
    numbersep=5pt,                  
    showspaces=false,                
    showstringspaces=false,
    showtabs=false,                  
    tabsize=2
}

\lstset{style=mystyle, language=Python2}

\begin{document}

\title{Operations for Relationships}

\maketitle

\pagenumbering{arabic}

\section{Encoding/Decoding Input Assumptions in JSON}


\subsection{In Program after analysis finished}

In reality we have assumption like $N - 1 - len_{values} \leq 0$ that need to be in a format that is more human readable.\\
Idea: move a negative variable to other side $N - 1 \leq len_{values}$ or any variable else.\\
\\
$[(N, (int), \_, {N: 1}),$\\
$(len_{values}, (int), N-1 < len_{values}, \{len_{values}: 0, N: 1\})$\\
$\times [(values, (int), \_, \{values: 0, N: 1\})]\ with\ delimiter\ "\ "]$\\
\begin{itemize}
  \item MultiInputAssumptionLattice 1 x
  \begin{itemize}
    \item InputAssumptionLattice 
    \begin{itemize}
      \item var name: N
      \item AssumptionLattice (int)
      \item relations: []
      \item input info: N: 1
    \end{itemize}
    \item MultiInputAssumptionLattice (with delimiter " ")
    \begin{itemize}
      \item InputAssumptionLattice 
      \begin{itemize}
	\item var name: $len_{values}$
	\item AssumptionLattice (int)
	\item relations: $[0 < len_{values}, N-1 < len_{values}]$
	\item input info: $\{len_{values}: 0, N: 1\}$
      \end{itemize}
      \item InputAssumptionLattice 
      \begin{itemize}
	\item var name: values
	\item AssumptionLattice (int)
	\item relations: []
	\item input info: $\{values: 0, N: 1\}$
      \end{itemize}
    \end{itemize}
  \end{itemize}
\end{itemize}

Important: You need to distinguish if there have to be at least 0 elements or if the assumptions just not hold if there are no elements or the number of elements is $\leq$ 0 .\\
Example1: for i in range(N): some assumptions\\
Example2: if len(values) $\leq$ 0: raise Error\\
To do so you have to make sure not to include i $leq$ 0 when analyzing a loop header.

\subsection{In Program after initial processing:}

$[((int, \_),$\\
$((int), [.IN[1]-1 < THIS.len])$\\
$\times [((int), \_)]\ wD\ " "]]$\\

\begin{itemize}
  \item List 
  \begin{itemize}
    \item Tuple 
    \begin{itemize}
      \item (int)
      \item $[]$
    \end{itemize}
    \item MultiInput with delimiter " "
    \begin{itemize}
      \item Tuple
      \begin{itemize}
	\item (int)
	\item $[.IN[1]-1 < THIS.len]$
      \end{itemize}
      \item Tuple
      \begin{itemize}
	\item (int)
	\item $[]$
      \end{itemize}
    \end{itemize}
  \end{itemize}
\end{itemize}


\subsection{Information needed for JSON}

before:\\
$[(N, (int), \_, {N: 1}),$\\
$(len_{values}, (int), N-1 < len_{values}, \{len_{values}: 0, N: 1\})$\\
$\times [(values, (int), \_, \{values: 0, N: 1\})]\ wD\ " "]$\\
after:\\
$[((int, \_),$\\
$((int), [.IN[1]-1 < THIS.len])$\\
$\times [((int), \_)]\ wD\ " "]]$\\
\\
A relationship is always stored at the second input that is read. As soon as we encounter a relationship the first value should already be in the list of saved values.
Encoded information:\\
\begin{itemize}
  \item assumptions per line:\\
  $[line1, line2, ...]$
  \item assumptions about repetition of lines:\\
  $N \times [line1, line2, ...]$
  \item assumptions on same line with delimiter:\\
  $len \times [assmps]\ with\ delimiter\ ","$
  \item assumptions about type:\\
  $((type), ...)$
  \item assumptions about relationship to current value:\\
  $(..., THIS < .IN[0])$
  \item assumptions about repetition number:\\
  $((type), relations) \times [assmps]$
\end{itemize}

\subsection{JSON Design}

\begin{itemize}
  \item TYPE: "bool" or "int" or "float" or "any"
  \item current element reference: "THIS"
  \item other element reference. id:number
  \item RELATION: left hand side of $\leq$ (example: $THIS - 1 + id{:}2$ of $THIS - 1 + id{:}2 \leq 0$. Input checker can then split by " ", look for "THIS" and a constant and/or an id)
  \item ASSUMPTION: $\{id: CONSTANT, type: TYPE, relations: [RELATION, ...]\}$
  \item DELIMITER: delimiter of values on the same line
  \item assumption with repetition:\\
  $\{repetition:ASSUMPTION/CONSTANT,$\\
  $assumptions:[ASSUMPTION, ...]$\\
  $(, delimiter:DELIMITER)\}$ (if delimiter exists assumptions hold for values on the same line, otherwise the values are on seperate lines)
  \item list of assumptions with identifiers: $[..., 0{:}\{ASSUMPTION(WITHREP)\}]$
\end{itemize}

\section{Checker Tasks}

\begin{enumerate}
  \item Read assumptions from JSON
  \begin{enumerate}
    \item Keep track of all ids whose values need to be stored for a later comparison. Every time a relationship is read save the value of the smaller id. That is the id of the value that has to be stored.
  \end{enumerate}
  \item Input: [assumptions]
  \item iterate through assumptions
  \begin{enumerate}
    \item Keep track of the next id that has to be stored (then we don't have to check for every value if it's in the list but just compare it to the next id that has to be saved).
    \item If a relationship is encountered 
  \end{enumerate}
\end{enumerate}


\end{document}















