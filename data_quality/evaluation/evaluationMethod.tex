\documentclass[11pt]{article}
\usepackage[a4paper, portrait, margin=1in]{geometry}
\usepackage{fancyhdr}
\usepackage{hyperref}
\usepackage{graphicx}
\usepackage{listings}
\usepackage{mathtools}
\usepackage{color}
\usepackage[dvipsnames]{xcolor}
\usepackage{float}
\usepackage[font={small,it}]{caption}

\usepackage{fancyhdr}
%\pagestyle{fancy}
\fancyhf{}
\renewcommand{\headrulewidth}{0pt}
\renewcommand{\footrulewidth}{0pt}
\newcommand{\bigcell}[2]{\begin{tabular}{@{}#1@{}}#2\end{tabular}}
 
\lstdefinelanguage{Python2}{
  language     = Python,
  morekeywords = {assert},
}
 
\lstdefinestyle{mystyle}{
    backgroundcolor=\color{white},   
    commentstyle=\color{ForestGreen},
    keywordstyle=\color{blue},
    numberstyle=\tiny\color{black},
    stringstyle=\color{red},
    basicstyle=\footnotesize,
    breakatwhitespace=false,         
    breaklines=true,                 
    captionpos=b,                    
    keepspaces=true,                 
    numbers=left,                    
    numbersep=5pt,                  
    showspaces=false,                
    showstringspaces=false,
    showtabs=false,                  
    tabsize=2
}

\lstset{style=mystyle, language=Python2}

\begin{document}

\title{Evaluation}

\maketitle

\pagenumbering{arabic}

\section{Evaluation Kinds}

\subsection{Evaluation of Analyzer}

\begin{enumerate}
    \item Choose example programs
    \item Evaluate strongest assumption by hand
    \item Show assumption found by analysis
    \item Compare
\end{enumerate}

\begin{itemize}
  \item \textit{What is the goal of the evaluation?} To show how powerful the analysis is. Show what the analysis is able to do and what not.
  \item \textit{What do you want to be able to decide after the evaluation?} If it makes sense to further investigate this idea of analysis.
  \item \textit{What is the audience of the evaluation?} Developers that want to use or improve the idea.
  \item \textit{What kind of information is needed to make a conclusion?} What the analysis can and cannot do.
  \item \textit{From what sources will the programs be collected?} Online sources
\end{itemize}

\subsection{Evaluation of Input Checker}

\subsubsection{User Study}

\begin{itemize}
    \item print Questionnaire Before, Scenario1+2, Task Explanation1+2, Questionnaire1+2
    \item bring mouse
    \item bring additional laptop to continue working on thesis
\end{itemize}


\begin{enumerate}
    \item Choose example programs, introduce errors in their input data
    \item Explain people what program does (what are inputs, what will be the output), introduce them to the scenario (want to run a program, have some input data, don't know the format of the input data)
    \item First without tool:
    \begin{enumerate}
        \item should try to correct input value by looking at error message
        \item Ask questions would they would like so that it is easiert (without user seeing our solution, what kind of solution would they expect?)
    \end{enumerate}
    \item With the tool:
    \begin{enumerate}
        \item should try to correct input value by using the tool
        \item Ask questions if they think the tool helps to fix the inputs. Ask what they think is good and what they would improve.
    \end{enumerate}
\end{enumerate}

Possible Questions:
\begin{itemize}
    \item Without tool: Want to show the need of a tool and the expectations people have for such a tool.
    \begin{itemize}
        \item What makes it difficult for you to correct the errors in the input data?
        \item What would make it easier for you to be able to find and fix errors in the input value?
        \item What kind of tool would you expect that helps you find the errors in the input data?
    \end{itemize}
    \item With tool: Want to show that our approach is a useful solution, want to know what improvements could be made.
    \begin{itemize}
        \item Do you think the tool is useful for fixing errors in the input data?
        \item Does the tool meet your expectations of a tool that should help you find and fix errors in the input data?
        \item What would you improve so that it is easier for you to reach your goal?
    \end{itemize}
\end{itemize}

\begin{itemize}
  \item \textit{What is the goal of the evaluation?} To show the reader how useful the tool is. Show that the tool has advantages over trying to solve problem without the tool.
  \item \textit{What do you want to be able to decide after the evaluation?} If the tool is useful and if it makes sense to further investigate this idea or to use this tool.
  \item \textit{What is the audience of the evaluation?} Developers (might want to follow up on this idea if it's useful) and users (might want to invest in this tool).
  \item \textit{What kind of information is needed to make a conclusion?} Comments from people that use the tool.
  \item \textit{From what sources will the programs be collected?} Online sources
\end{itemize}



\end{document}

