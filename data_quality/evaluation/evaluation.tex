\documentclass[11pt]{article}
\usepackage[a4paper, portrait, margin=1in]{geometry}
\usepackage{fancyhdr}
\usepackage{hyperref}
\usepackage{graphicx}
\usepackage{listings}
\usepackage{mathtools}
\usepackage{color}
\usepackage[dvipsnames]{xcolor}
\usepackage{float}
\usepackage[font={small,it}]{caption}

\usepackage{fancyhdr}
%\pagestyle{fancy}
\fancyhf{}
\renewcommand{\headrulewidth}{0pt}
\renewcommand{\footrulewidth}{0pt}
\newcommand{\bigcell}[2]{\begin{tabular}{@{}#1@{}}#2\end{tabular}}
 
\lstdefinelanguage{Python2}{
  language     = Python,
  morekeywords = {assert},
}
 
\lstdefinestyle{mystyle}{
    backgroundcolor=\color{white},   
    commentstyle=\color{ForestGreen},
    keywordstyle=\color{blue},
    numberstyle=\tiny\color{black},
    stringstyle=\color{red},
    basicstyle=\footnotesize,
    breakatwhitespace=false,         
    breaklines=true,                 
    captionpos=b,                    
    keepspaces=true,                 
    numbers=left,                    
    numbersep=5pt,                  
    showspaces=false,                
    showstringspaces=false,
    showtabs=false,                  
    tabsize=2
}

\lstset{style=mystyle, language=Python2}

\begin{document}

\title{Evaluation}

\maketitle

\pagenumbering{arabic}

\section{Questions}

% https://managementhelp.org/evaluation/program-evaluation-guide.htm#anchor1575679

\begin{itemize}
  \item What is the goal of the evaluation? What do you want to be able to decide after the evaluation?
  \item What is the audience of the evaluation?
  \item What kind of information is needed to make a conclusion?
  \item From what sources will the programs be collected?
\end{itemize}

\section{Goal of evaluation}

\begin{itemize}
  \item Show what information the analysis can gather and what not.
  \begin{itemize}
    \item Find suitable examples.
    \item Show based on the example how many errors are found and how many not.\\
    How to show the amount of errors that are not found?\\
    Idea1: Run random inputs on the program and compare what errors are not catched by the input checker.
  \end{itemize}
  \item Show what errors the input checker can report and what not.
  \begin{itemize}
    \item Find suitable examples.
  \end{itemize}
  \item Show how the input checker makes it easier for the user to find and correct wrong inputs.
  \begin{itemize}
    \item Show the advantage of just giving the user all the assumptions found by the analysis.
  \end{itemize}
\end{itemize}


\section{Showing errors found/missed}

\begin{enumerate}
  \item Choose a program.
  \item Construct an input (or take from web) that yields no errors.
  \item Create a number of possible modification like add a value, remove a value, change a value (type, range), remove delimiter, add empty values, remove lines... Maybe find some moditifaction ideas from common errors people make in input data.
  \item Apply these modifications (single or in combination) and check if program yields an error and check if the input checker finds a problem.
  \item Count the amount of errors found/missed.
\end{enumerate}


\end{document}

