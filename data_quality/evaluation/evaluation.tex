\documentclass[11pt]{article}
\usepackage[a4paper, portrait, margin=1in]{geometry}
\usepackage{fancyhdr}
\usepackage{hyperref}
\usepackage{graphicx}
\usepackage{listings}
\usepackage{mathtools}
\usepackage{color}
\usepackage[dvipsnames]{xcolor}
\usepackage{float}
\usepackage[font={small,it}]{caption}

\usepackage{fancyhdr}
%\pagestyle{fancy}
\fancyhf{}
\renewcommand{\headrulewidth}{0pt}
\renewcommand{\footrulewidth}{0pt}
\newcommand{\bigcell}[2]{\begin{tabular}{@{}#1@{}}#2\end{tabular}}
 
\lstdefinelanguage{Python2}{
  language     = Python,
  morekeywords = {assert},
}
 
\lstdefinestyle{mystyle}{
    backgroundcolor=\color{white},   
    commentstyle=\color{ForestGreen},
    keywordstyle=\color{blue},
    numberstyle=\tiny\color{black},
    stringstyle=\color{red},
    basicstyle=\footnotesize,
    breakatwhitespace=false,         
    breaklines=true,                 
    captionpos=b,                    
    keepspaces=true,                 
    numbers=left,                    
    numbersep=5pt,                  
    showspaces=false,                
    showstringspaces=false,
    showtabs=false,                  
    tabsize=2
}

\lstset{style=mystyle, language=Python2}

\begin{document}

\title{Evaluation}

\maketitle

\pagenumbering{arabic}

\section{Evaluation Kinds}

\subsection{Evaluation of Analyzer}

We choose programs. The resulting precondition will be found by hand and we compare the result of the analysis with that.

\begin{itemize}
  \item \textit{What is the goal of the evaluation?} To show how powerful the analysis is. Show what the analysis is able to do and what not. Show how the analysis can be extended to make it more powerful.
  \item \textit{What do you want to be able to decide after the evaluation?} If it makes sense to further investigate this idea of analysis.
  \item \textit{What is the audience of the evaluation?} Developers that want to use or improve the idea.
  \item \textit{What kind of information is needed to make a conclusion?} What the analysis can and cannot do. How can the analysis be extended and what is not possible.
  \item \textit{From what sources will the programs be collected?} Online sources
\end{itemize}

\subsection{Evaluation of Input Checker}

We are doing a user study. Four people (two with programmin experience, two without). Two will try to solve problems with input data without the tool. Want to gather information how difficult it is and what they think would help them. The others will work with the tool and give feedback.

\begin{itemize}
  \item \textit{What is the goal of the evaluation?} To show the reader how useful the tool is. Show that the tool has advantages over trying to solve problem without the tool.
  \item \textit{What do you want to be able to decide after the evaluation?} If the tool is useful and if it makes sense to further investigate this idea or to use this tool.
  \item \textit{What is the audience of the evaluation?} Developers (might want to follow up on this idea if it's useful) and users (might want to invest in this tool).
  \item \textit{What kind of information is needed to make a conclusion?} Comments from people that use the tool.
  \item \textit{From what sources will the programs be collected?} Online sources
\end{itemize}

\section{Questions}

% https://managementhelp.org/evaluation/program-evaluation-guide.htm#anchor1575679

\begin{itemize}
  \item What is the goal of the evaluation?
  \begin{itemize}
    \item Show for what kind of programs assumptions will be found.
    \item Show what assumptions are found for programs
    \item Show how useful the program is to correct input errors
  \end{itemize}
  \item What do you want to be able to decide after the evaluation?
  \begin{itemize}
    \item Is this tool useful to check for errors in the input data?
    \item Is this tool powerful enough to find errors in the input data?
    \item Is a user (with/without experience in computer science) able to correct issues in the data?
    \item Does the tool help a user to find and fix problems in the input data?
  \end{itemize}
  \item What is the audience of the evaluation?
  \item What kind of information is needed to make a conclusion?
  \item From what sources will the programs be collected?
\end{itemize}

\section{Goal of evaluation}

\begin{itemize}
  \item Show what assumptions the analysis can gather
  \begin{itemize}
    \item For every assumption the analysis can gather automatically generate simple programs to showcase what kind of programs can successfully be checked.
  \end{itemize}
  \item Show what information the analysis can gather and what not.
  \begin{itemize}
    \item Find suitable examples.
    \item Show based on the example how many errors are found and how many not.\\
    How to show the amount of errors that are not found?\\
    Idea1: Run random inputs on the program and compare what errors are not catched by the input checker.
  \end{itemize}
  \item Show what errors the input checker can report and what not.
  \begin{itemize}
    \item Find suitable examples.
  \end{itemize}
  \item Show how the input checker makes it easier for the user to find and correct wrong inputs.
  \begin{itemize}
    \item Show the advantage of just giving the user all the assumptions found by the analysis.
  \end{itemize}
\end{itemize}


\section{Showing errors found/missed}

\begin{enumerate}
  \item Choose a program.
  \item Construct an input (or take from web) that yields no errors.
  \item Create a number of possible modification like add a value, remove a value, change a value (type, range), remove delimiter, add empty values, remove lines... Maybe find some moditifaction ideas from common errors people make in input data.
  \item Apply these modifications (single or in combination) and check if program yields an error and check if the input checker finds a problem.
  \item Count the amount of errors found/missed.
\end{enumerate}


\end{document}

