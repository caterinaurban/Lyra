\documentclass[11pt]{article}
\usepackage[a4paper, portrait, margin=1in]{geometry}
\usepackage{fancyhdr}
\usepackage{hyperref}
\usepackage{graphicx}
\usepackage{listings}
\usepackage{mathtools}
\usepackage{color}
\usepackage[dvipsnames]{xcolor}
\usepackage{float}% If comment this, figure moves to Page 2
% Fancy header package for version number
\usepackage[font={small,it}]{caption}

\usepackage{fancyhdr}
%\pagestyle{fancy}
\fancyhf{}
\renewcommand{\headrulewidth}{0pt}
\renewcommand{\footrulewidth}{0pt}
\newcommand{\bigcell}[2]{\begin{tabular}{@{}#1@{}}#2\end{tabular}}
 
\lstdefinelanguage{Python2}{
  language     = Python,
  morekeywords = {assert},
}
 
\lstdefinestyle{mystyle}{
    backgroundcolor=\color{white},   
    commentstyle=\color{ForestGreen},
    keywordstyle=\color{blue},
    numberstyle=\tiny\color{black},
    stringstyle=\color{red},
    basicstyle=\footnotesize,
    breakatwhitespace=false,         
    breaklines=true,                 
    captionpos=b,                    
    keepspaces=true,                 
    numbers=left,                    
    numbersep=5pt,                  
    showspaces=false,                
    showstringspaces=false,
    showtabs=false,                  
    tabsize=2
}

\lstset{style=mystyle, language=Python2}

\begin{document}

\title{Code to List Abstraction}

\maketitle

\pagenumbering{arabic}

\section{Ideas}

\begin{itemize}
\item problem: how to automate/respresent how the lines look like? for example if we have a for loop that iterates over every file in line and we know that each line should look the same.

solution: also use a list abstraction for the vertical format of the file: \verb|{0} T {2} T {len}| means that there should at least be 2 lines =$>$ adjust algorithm design: first check if the number of lines is correct, because if not there might be a lot of errors if the lines are formatted incorrectly

\item problem: we have decision trees for values but we also kind of need decision trees to decide what the input should look like. so why not just abstract values as single element lists too and use that one decision tree
\end{itemize}

\section{Assignments}

every array starts with \verb|{0} T {len}?|

\begin{itemize}
\item \textit{x = a[2]}

array access \verb|==> a = {0} T {3} T {len}?|


\item \textit{x = a[i]}

array access \verb|==> a = {0} T {i}? T {i+1} T {len}?|


\item \textit{x = int(a[i])}

array access \verb|==> a = {0} T {i}? int {i+1} T {len}?|


\item \textit{x = a[i] \# with x $<$ 10}

array access \verb|==> a = {0} T {i}? <10 {i+1} T {len}?|


\item \textit{a = in.split(',')}

in.split(',') \verb|==> in = {0} T {len}? with delimiter ','|


\item \textit{a, b = in.split(',')}

in.split(',') \verb|==> in = {0} T {1} T {2} with delimiter ','|


\item \textit{in = input()}

array access \verb|==> LINES = {0} T {len-1}? ({0} T {len}? with delimiter ',') {len}|


\section{Split (if)}

\item \textit{if a[10] == x: y = a[11]}

before if: \verb|TOP|

inside if: \verb|a == {0} T {11} T {12} T {len}?|

else: \verb|TOP|

join: \verb|a == {0} T {10} ==x {11} T {len}?|
\begin{itemize}
\item True: \verb|a == {0} T {11} T {12} T {len}?|
\begin{itemize}
\item True: TOP
\item False: BOTTOM
\end{itemize}
\item False: TOP
\end{itemize}


\item \textit{if i $<$ 10: y = a[i]}

before if: \verb|TOP|

inside if: \verb|a == {0} T {i}? T {i+1} T {len}?|

else: \verb|TOP|

join: \verb|i < 10|
\begin{itemize}
\item True: \verb|a == {0} T {i}? T {i+1} T {len}|
\begin{itemize}
\item True: TOP
\item False: BOTTOM
\end{itemize}
\item False: TOP
\end{itemize}


\item \textit{if i $<$ 10: i = 11; y = a[i]}

before if: \verb|TOP|

inside if: \verb|a == {0} T {i}? T {i+1} T {len} ==> a == {0} T {11} T {12} T {len}|

else: \verb|TOP|

join: \verb|i < 10|
\begin{itemize}
\item True: \verb|a == {0} T {11} T {12} T {len}|
\begin{itemize}
\item True: TOP
\item False: BOTTOM
\end{itemize}
\item False: TOP
\end{itemize}


\item \textit{if i $<$ 0: i = 10; end; y = a[i]}

before if: \verb|a == {0} T {i}? T {i+1} T {len}?|

inside if: \verb|a == {0} T {10}? T {11} T {len}?|

else: \verb|a == {0} T {i}? T {i+1} T {len}?|

join: \verb|i < 0|
\begin{itemize}
\item True: \verb|a == {0} T {10}? T {11} T {len}?|
\begin{itemize}
\item True: TOP
\item False: BOTTOM
\end{itemize}
\item False: \verb|a == {0} T {i}? T {i+1} T {len}?|
\begin{itemize}
\item True: TOP
\item False: BOTTOM
\end{itemize}
\end{itemize}


\item \textit{if i $<$ 0: a = input()}

before if: \verb|TOP|

inside if: \verb|LINES = {0} T {len-1}? T {len}|

else: \verb|TOP|

join: \verb|i < 00|
\begin{itemize}
\item True: \verb|LINES = {0} T {len-1}? T {len}|
\begin{itemize}
\item True: TOP
\item False: BOTTOM
\end{itemize}
\item False: TOP
\end{itemize}

\end{itemize}

\section{input}

after the analysis is done LINES has to be adjusted to start from 0: set the smalles bound after \{0\} to 0 and solve for len, then compute bounds:

\verb|LINES = {0} T {len-3} y {len-1}? x {len}|

with len-3 == 0 ==$>$ len == 3

\verb|==> LINES = {0} y {2} x {3}|

\section{joins}

\begin{itemize}

\item \textit{join \{a\} x \{b\} and \{a\} y \{b\}}

\verb|{a} x U y {b}|

example: \verb|{0} >0 {10} JOIN {0} >1 {10} ==> {0} >0 {10}|

results in information loss => better use decision tree


\item \textit{join \{a\} x1 \{i\} x2 \{b\} and \{a\} y \{b\}}

\verb|{a} x1 U y {i U b U b}|

example: \verb|{0} >0 {10} >8 {20} JOIN {0} >1 {20} ==> {0} >0 {10} >1 {20}|

other solution would be to use decision tree: 

\verb|cond1 --T--> {0} >0 {10} >8 {20}, cond1 --F--> {0} >1 {20}|


\item \textit{join \{a\} x1 \{i\} x2 \{b\} and \{a\} y1 \{j\} y2 \{b\}}

--- simple solution:

\verb|{a} x1 U y1 ...|

\verb|{i} x2 {b} JOIN {j} y2 {b}|

\verb|{a} x1 U y1 U x2 {b} JOIN {a} x1 U y2 U y2 {b}|

\verb|{a} x1 U y1 U x2 U y2 {b}|

example: \verb|{0} >0 {10} >8 {20} JOIN {0} >1 {15} >7 {20} ==> {0} >0 {20}|

--- more advanced solution:

\verb|{a} x1 U y1 ...|

\verb|{i} x2 {b} JOIN {j} y2 {b}|

if i $<$ j: 

\verb|{i} x2 {j} x2 {b} JOIN {i} y1 {j} y2 {b}|

\verb|{a} x1 U y1 {i} x2 U y1 {j} x2 U y2 {b}|

example: \verb|{0} >0 {10} >8 {20} JOIN {0} >1 {15} >7 {20} ==> {0} >0 {10} >1 {15} >7 {20}|

--- or else decision tree


\item \textit{start or end do not match}

no unification


\item \textit{joining with len}

all elements have same condition

\verb|{0} >1 {len} JOIN {0} >2 {len} ==> {0} >1 {len}|

both arrays are splitted at same location:

\verb|{0} >1 {5} >3 {len} JOIN {0} >2 {5} >1 {len} ==> {0} >1 {5} >1 {len}|

one or both arrays are splitted:

\verb|{0} >1 {i} >3 {len} JOIN {0} >2 {len} ==> ?|

len is the actual length of the array. After a split the length of the array might depend on a condition so we cannot join if we aren't sure that the lengths are equal. Problem: \verb|{0} >1 {5} >3 {len} JOIN {0} >2 {len} =/=> {0} >1 {5} >2 {len} or {0} >1 {[5,len]}| because in the second case it doesn't have to hold that $len > 5$. Therefore the best solution might be using a decision tree.


\item \textit{Question: join or decision tree?}

example: \verb|if x > 10: y = a[5] # y>1; else: y = a[5] # y>2|

join: \verb|a = {0} T {5} >1 {6} T {len}| (simpler, less exact, not always possible)

decision tree: \verb|x > 10 --T--> a = {0} T {5} >1 {6} T {len}|

\verb|x > 10 --F--> a = {0} T {5} >2 {6} T {len}| (more complex, more exact, always possible)


\item \textit {double if example}

QUESTION: better to have simple constraints and then later add them to list abstraction or to always use list abstraction?

ANSWER: we either have conditions like (1) a = [(3,$>$3),(4,T),(6,T),(len,Nil,?)] or (2) [a(3) $>$ 3, len(a) $>=$ 6]. Version (2) would just fill up with conditions, maybe some unifications in between and then in the end we will create our list abstraction so that the Checker has the assumptions per line. Using Version (1) might be more difficult during analysis but in the end we already have our list abstraction. Furthermore using Version (2) will result in the exact same list abstraction as Version (1).

\begin{lstlisting}[numbers=none]
if a[3] > 2:
    y = a[6]
if x > 10:
    y = a[5]
else:
    if a[2] > 10:
        y = 1/a[2]
\end{lstlisting}

--- final compressed:

\verb|a == {0} T {3} >2 {4} T {6} T {7} T {len}?|

\verb|OR a = {0} T {3} <=2 {4} T {len}?|

\verb|AND (x > 10 AND a == {0} T {5} T {6} T {len}?|

\verb|OR x <= 10 AND a == {0} T {2} >10, !=0 {3} T {len}?|

\verb|OR x <= 10 AND a == {0} T {2} <=10 {3} T {len}?)|

--- final:

\verb|a == {0} T {3} >2 {4} T {len}? AND a == {0} T {6} T {7} T {len}?|

\verb|OR a == {0} T {3} <=2 {4} T {len}?|

\verb|AND (x > 10 AND a == {0} T {5} T {6} T {len}?|

\verb|OR x <= 10 AND a == {0} T {2} >10 {3} T {len}? AND a == {0} T {2} !=0 {3} T {len}?|

\verb|OR x <= 10 AND a == {0} T {2} <=10 {3} T {len}?)|

--- decision tree:

\verb|a == {0} T {3} >2 {4} T {len}?|
\begin{itemize}
\item True: \verb|a == {0} T {6} T {7} T {len}?|
\begin{itemize}
\item True: TOP
\item False: BOTTOM
\end{itemize}
\item False: TOP
\end{itemize}

\verb|x > 10|
\begin{itemize}
\item True: \verb|a == {0} T {5} T {6} T {len}?|
\begin{itemize}
\item True: TOP
\item False: BOTTOM
\end{itemize}
\item False: \verb|a == {0} T {2} >10 {3} T {len}?|
\begin{itemize}
\item True: \verb|a == {0} T {2} !=0 {3} T {len}?|
\begin{itemize}
\item True: TOP
\item False: BOTTOM
\end{itemize}
\item False: TOP
\end{itemize}
\end{itemize}

--- final2:

\verb|a(3) > 2 AND len(a) > 6 OR a(3) <= 2|

\verb|AND (x > 10 AND len(a) > 5 OR x <= 10 AND a(2) > 10 AND a(2) != 0|

\verb|OR x <= 10 AND a(2) <= 10)|

final list constraints:

\verb|a == {0} T {3} >2 {4} T {7} T {len}? OR a == {0} T {3} <=2 {4} T {len}?|

\verb|AND (x > 10 AND a == {0} T {6} T {len}? OR a =={0} T {2} >10, !=0 {3} T {len}?|

\verb|OR x <= 10 AND a =={0} T {2} <= 10 {3} T {len}?)|

--- decision tree v2:

\verb|a(3) > 2|
\begin{itemize}
\item True: \verb|len(a) > 6|
\begin{itemize}
\item True: TOP
\item False: BOTTOM
\end{itemize}
\item False: TOP
\end{itemize}

\verb|x > 10|
\begin{itemize}
\item True: \verb|len(a) > 5|
\begin{itemize}
\item True: TOP
\item False: BOTTOM
\end{itemize}
\item False: \verb|a(2) > 10|
\begin{itemize}
\item True: \verb|a(2) != 0|
\begin{itemize}
\item True: TOP
\item False: BOTTOM
\end{itemize}
\item False: TOP
\end{itemize}
\end{itemize}

\item \textit{if with input example}

\begin{lstlisting}[numbers=none]
x = input()
if x > 10:
    a = input()
\end{lstlisting}

\verb|LINES == {0} T {len-2} >10 {len-1} T {len}|
\begin{itemize}
\item True: \verb|LINES == {0} T {len-1} T {len}|
\begin{itemize}
\item True: TOP
\item False: BOTTOM
\end{itemize}
\item False: TOP
\end{itemize}

\verb|LINES == {0} >10 {1} T {2}|
\begin{itemize}
\item True: \verb|LINES == {0} T {1} T {2}|
\begin{itemize}
\item True: TOP
\item False: BOTTOM
\end{itemize}
\item False: TOP
\end{itemize}


\end{itemize}

\section{join (while)}

\begin{itemize}

\item \textit{while loop with loop unrolling}

PROBLEM: if we exit the loop with \verb|{0} T {i}? int {i+1} int {i+2} int {len}?| and then get to \verb|i = 0| we end up with \verb|{0} int {1} int {2} int {len}?|. What happens now if the input only has 1 element? We will output an error saying 'There must be at least 2 elements' which might just not be true because we did too much loop unrolling.

\item \textit{while loop using information about the range of i (with condition i $<$ len(a))}

will only work if we are sure to loose information of i as soon as assumption does not hold for EVERY i in range (otherwise will get wrong assumptions)

\begin{lstlisting}[numbers=none]
i = 0
while i < len(a):
    x = int(a[i])
    i += 1
\end{lstlisting}

loop after \verb|==> a = {0} T {len}? and i: [len(a), inf]|

loop end \verb|==> a = {0} T {len}? and i: [len(a), len(a)]|

loop[i+=1] \verb|==> a = {0} T {len}? and i: [len(a)-1, len(a)-1]|

loop[x=int(a[i])] \verb|==> a = {0} T {i}? int {i+1} T {len}? and i: [len(a)-1, len(a)-1]|

loop end \verb|==> a = {0} T {[i, len]}? and i: [len(a)-1, len(a)]|

loop[i+=1] \verb|==> a = {0} T {[i+1, len]}? and i: [len(a)-2, len(a)-1]|

loop[x=int(a[i])] \verb|==> a = {0} T {i}? int {i+1} T {[i+1, len]}? and i: [len(a)-2, len(a)-1]|

loop end \verb|==> a = {0} T {[i+1, len]}? and i: [len(a)-2, len(a)]|

loop[i+=1] \verb|==> a = {0} T {[i+1,len]}? and i: [len(a)-3, len(a)-1]|

loop[x=int(a[i])] \verb|==> a = {0} T {i}? int {i+1} T {[i+1,len]}? and i: [-inf, len(a)-1]|

loop before \verb|==> i < N --T-->  a = {0} T {i}? int {i+1} T {[i+1,len]}? and i: [-inf, len(a)-1]| \verb|==> i < N --F-->  a = {0} T {len}?|

loop before[i=0] \verb|==> 0 < N --T-->  a = {0} int {len(a)} T {[len(a),len]}?| \verb|==> 0 < N --F-->  a = {0} T {len}?|


\item \textit{while loop using information about the range of i (with condition i $<$ N)}

IDEA: try to find fix point and keep information about access variable as long as i is continuous. as soon as for example i=i+2 we cannot show it as a range and just loose the information about i. so something like a SAFE\_JOIN

\begin{lstlisting}[numbers=none]
i = 0
while i < N:
    x = int(a[i])
    i += 1
\end{lstlisting}

loop after \verb|==> a = {0} T {len}? and i: [N, inf]|

loop end \verb|==> a = {0} T {len}? and i: [0, N-1]|

loop[i+=1] \verb|==> a = {0} T {len}? and i: [0, N-1]|

loop[x=int(a[i])] \verb|==> a = {0} T {i}? int {i+1} T {len}? and i: [0, N-1]|

loop end \verb|==> a = {0} T {[i,len]}? and i: [0, N-1]|

loop[i+=1] \verb|==> a = {0} T {[i+1,len]}? and i: [0, N-1]|

loop[x=int(a[i])] \verb|==> a = {0} T {i}? int {i+1} T {[i+1,len]}? and i: [0, N-1]|

loop end \verb|==> a = {0} T {[i,len]}? and i: [0, N-1]|

loop[i+=1] \verb|==> a = {0} T {[i+1,len]}? and i: [0, N-1]|

loop[x=int(a[i])] \verb|==> a = {0} T {i}? int {i+1} T {[i+1,len]}? and i: [0, N-1]|

loop before \verb|==> i < N --T-->  a = {0} T {i}? int {i+1} T {[i+1,len]}? and i: [0, N-1]| \verb|==> i < N --F-->  a = {0} T {len}?|

loop before[i=0] \verb|==> 0 < N --T-->  a = {0} int {N-1} int {N} T {[N,len]}?| \verb|==> 0 < N --F-->  a = {0} T {len}?|

-- does not work: cannot assume that i is all values between [0,N]

loop after \verb|==> a = {0} T {len}? and i: [N, inf]|

loop end \verb|==> a = {0} T {len}? and i: [0, N]|

loop[i+=1] \verb|==> a = {0} T {len}? and i: [0, N-1]|

loop[x=int(a[i])] \verb|==> a = {0} T {i}? int {i+1} T {len}? and i: [0, N-1]|

loop end \verb|==> a = {0} T {[i,len]}? and i: [0, N]|

loop[i+=1] \verb|==> a = {0} T {[i+1,len]}? and i: [0, N-1]|

loop[x=int(a[i])] \verb|==> a = {0} T {i}? int {i+1} T {len}? and i: [0, N-1]|

loop before \verb|==> i < N --T-->  a = {0} T {i}? int {i+1} T {len}? and i: [0, N-1]| 

\verb|==> i < N --F-->  a = {0} T {len}?|

loop before[i=0] \verb|==> 0 < N --T-->  a = {0} int {N-1} int {N} T {len}?| 

\verb|==> 0 < N --F-->  a = {0} T {len}?|

--- directly with information about i:

loop after \verb|==> a = {0} T {len}? and i: [N, inf]|

loop end \verb|==> a = {0} T {len}? and i: [0, N-1]|

loop[i+=1] \verb|==> a = {0} T {len}? and i: [0, N-1]|

loop[x=int(a[i])] \verb|==> a = {0} int {N} T {len}? and i: [0, N-1]|

loop end \verb|==> a = {0} T {len}? and i: [0, N-1]|

loop[i+=1] \verb|==> a = {0} T {len}? and i: [0, N-1]|

loop[x=int(a[i])] \verb|==> a = {0} int {N} T {len}? and i: [0, N-1]|

loop before \verb|==> i < N --T-->  a = {0} int {N} T {len}?| 

\verb|==> i < N --F-->  a = {0} T {len}?|

loop before[i=0] \verb|==> 0 < N --T-->  a = {0} int {N} T {len}?| 

\verb|==> 0 < N --F-->  a = {0} T {len}?|


\item \textit{while loop with fix point iteration using information about the range of i (with loop steps of 2}

\begin{lstlisting}[numbers=none]
i = 0
while i < N:
    x = int(a[i])
    i += 2
\end{lstlisting}

loop after \verb|==> a = {0} T {len}? and i: [N, inf]|

loop end \verb|==> a = {0} T {len}? and i: [0, N-1]|

loop[i+=2] \verb|==> a = {0} T {len}? and i: [0, N-3]|

loop[x=int(a[i])] \verb|==> a = {0} T {i}? int {i+1} T {len}? and i: [0, N-3]|

loop end \verb|==> a = {0} T {i}? int {i+1} T {len}? and i: [0, N-3]|

loop[i+=2] \verb|==> a = {0} T {i+2}? int {i+3} T {len}? and i: [0, N-1]|

loop[x=int(a[i])] \verb|==> a = {0} T {i}? int {i+1} T {[i+1,len]}? and i: [0, N-1]|

loop end \verb|==> a = {0} T {[i,len]}? and i: [0, N-1]|

loop[i+=2] \verb|==> a = {0} T {[i+2,len]}? and i: [0, N-1]|

loop[x=int(a[i])] \verb|==> a = {0} T {i}? int {i+1} T {[i+2,len]}? and i: [0, N-1]|

...


\item \textit{while loop with loop unrolling using information about the range of i (with loop steps of 2}

\begin{lstlisting}[numbers=none]
i = 0
while i < N:
    x = int(a[i])
    i += 2
\end{lstlisting}

loop after \verb|==> a = {0} T {len}? and i: [N, inf]|

loop end \verb|==> a = {0} T {len}? and i: [0, N-1]|

loop[i+=2] \verb|==> a = {0} T {len}? and i: [0, N-3]|

loop[x=int(a[i])] \verb|==> a = {0} T {i}? int {i+1} T {len}? and i: [0, N-3]|

loop end \verb|==> a = {0} T {i}? int {i+1} T {len}? and i: [0, N-3]|

loop[i+=2] \verb|==> a = {0} T {i+2}? int {i+3} T {len}? and i: [0, N-1]|

loop[x=int(a[i])] \verb|==> a = {0} T {i}? int {i+1} T {i+2} int {i+3} T {len}? and i: [0, N-1]|

loop before \verb|==> i < N --T--> a = {0} T {i}? int {i+1} T {i+2} int {i+3} T {len}?| 

\verb|==> i < N --F-->  a = {0} T {len}?|

loop before[i=0] \verb|==> 0 < N --T-->  a = {0} int {1} T {2} int {3} T {len}?| 

\verb|==> 0 < N --F-->  a = {0} T {len}?|

PROBLEM: if len(a) $<$ 3

\end{itemize}

\section{Input in While Loop}


\begin{itemize}

\item PROBLEM: need to know how many loops there will be, otherwise we don't know how to match the input lines

\begin{lstlisting}[numbers=none]
i = 0
while i < N:
    a = input()
    x = a[5]
    i = i+1
\end{lstlisting}

loop after \verb|==> a = {0} T {len}? and i: [N, inf]|

loop end \verb|==> a = {0} T {len}? and i: [N, N]|

loop[i+=1] \verb|==> a = {0} T {len}? and i: [N-1, N-1]|

loop[x=a[5])] \verb|==> a = {0} T {5} int {6} T {len}? and i: [N-1, N-1]|

loop[a=input()] \verb|==> LINES = {0} T {len-1}? ({0} T {5} int {6} T {len}?) {len} and i: [N-1, N-1]|

loop end \verb|==> a = {0} T {len}? and i: [N-1, N]|

loop[i+=1] \verb|==> a = {0} T {len}? and i: [N-2, N-1]|

loop[x=a[5]] \verb|==> a = {0} T {5} int {6} T {len}? and i: [N-2, N-1]|

loop[a=input()] \verb|==> LINES = {0} T {len-2}? ({0} T {5} int {6} T {len}?) {len} and i: [N-2, N-1]|

loop end \verb|==> a = {0} T {len}? and i: [N-2, N]|

loop[i+=1] \verb|==> a = {0} T {len}? and i: [N-3, N-1]|

loop[x=a[5]] \verb|==> a = {0} T {5} int {6} T {len}? and i: [N-3, N-1]|

loop[a=input()] \verb|==> LINES = {0} T {len-2}? ({0} T {5} int {6} T {len}?) {len} and i: [-inf, N-1]|

loop before \verb|==> i < N --T-->  input() = {0} T {5} int {6} T {len}? for i in [-inf, N-1]| \verb|==> i < N --F-->  a = {0} T {len}?|

loop before[i=0] \verb|==> 0 < N --T-->  LINES = {0} T {len1-(N-1)} ({0} T {5} int {6} T {len2}?) {len1}| \verb|==> 0 < N --F-->  a = {0} T {len}?|


\item \textit{examples of how data is read in loop}

\begin{lstlisting}[numbers=none]
for i in range(T):
    x = input()
    
i = 0
while i < T:
    x = input()
    i = i+1
\end{lstlisting}

\begin{lstlisting}[numbers=none]
for line in inputfile:
    x = line
    
lines = inputfile.readlines()
i = 0
while i < len(lines):
    x = lines[i]
    i = i + 1
\end{lstlisting}
\end{itemize}
\end{document}





















