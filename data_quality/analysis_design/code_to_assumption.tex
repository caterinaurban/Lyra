\documentclass[11pt]{article}
\usepackage[a4paper, portrait, margin=1in]{geometry}
\usepackage{fancyhdr}
\usepackage{hyperref}
\usepackage{graphicx}
\usepackage{listings}
\usepackage{mathtools}
\usepackage{color}
\usepackage[dvipsnames]{xcolor}
\usepackage{float}% If comment this, figure moves to Page 2
% Fancy header package for version number
\usepackage[font={small,it}]{caption}

\usepackage{fancyhdr}
%\pagestyle{fancy}
\fancyhf{}
\renewcommand{\headrulewidth}{0pt}
\renewcommand{\footrulewidth}{0pt}
\newcommand{\bigcell}[2]{\begin{tabular}{@{}#1@{}}#2\end{tabular}}
 
\lstdefinelanguage{Python2}{
  language     = Python,
  morekeywords = {assert},
}
 
\lstdefinestyle{mystyle}{
    backgroundcolor=\color{white},   
    commentstyle=\color{ForestGreen},
    keywordstyle=\color{blue},
    numberstyle=\tiny\color{black},
    stringstyle=\color{red},
    basicstyle=\footnotesize,
    breakatwhitespace=false,         
    breaklines=true,                 
    captionpos=b,                    
    keepspaces=true,                 
    numbers=left,                    
    numbersep=5pt,                  
    showspaces=false,                
    showstringspaces=false,
    showtabs=false,                  
    tabsize=2
}

\lstset{style=mystyle, language=Python2}

\begin{document}

\pagenumbering{arabic}

\section{Numeric}

\begin{itemize}
\item \textit{x = items[2] \# with items being an array}

len(items) $>$ 2 or items = \verb|{0} T {3} T {len}?|


\item \textit{for i in range(N): \# with i $<$ len(array)}

N - 1 $<$ len(array) or array = \verb|{0} T {N} T {len}?|


\item \textit{y = 1 / x}

x != 0


\item \textit{x = array[i] \# with x != 0}

array = \verb|{0} x != 0 {i + 1} T {len}?|


\item \textit{for i in range(N): \# with array = \{0\} x != 0 \{i + 1\} T \{len\}}

array = \verb|{0} x != 0 {N} T {len}?|


\item \textit{x = array[i][j]}

array = \verb|{0} T {i}? ({0} T {j+1} T {len}?) {i+1} T {len}?|


\item \textit{for i in range(N): for j in range(M): ... array[i][j] ...}

iterationM1: array = \verb|{0} T {i}? ({0} T {j+1} T {lenA}?) {i+1} T {lenB}?|
             array = \verb|{0} T {N-1}? ({0} T {M}? T {lenA}?) {N} T {lenB}?|
iterationM2: array = \verb|{0} T {N-1}? ({0} T {M}? T {lenA}?) {N} T {lenB}?|
             array = \verb|{0} T {N-1}? ({0} T {M}? T {lenA}?) {N} T {lenB}?|
...
iterationN2: array = \verb|{0} T {N-2}? ({0} T {M} T {lenA}?) {N-1} T {lenB}?|
             array = \verb|{0} T {N-2}? ({0} T {M} T {lenA}?) {N} T {lenB}?|

\item \textit{raise Error()}

BOTTOM


\item \textit{if a == b: raise Error()}

a != b


\item start with bottom as long as no print() or write() statement

\end{itemize}

\section{Structure}

'line' is from input.

\begin{itemize}

\item \textit{x = items['chrom'] \# with items being a dictionary}

items contains 'chrom'

final regex with items = line.split(','): \verb|^(.*,)?chrom(,.*)?$|


\item \textit{items = line.split(',') \# with len(items) $>$ 2}

line contains at least two ',' $==>$ line is like \verb|^[^,]*,[^,]*,.*$|

final regex: \verb|^[^,]*,[^,]*,.*$|

with list abstraction: \verb|{0} [^,] {3} T {len}? with delimiter ','|


\item \textit{x = items[2].split(';')[3]}

same as: y = items[2].split(';'); x = y[3]

similar as before: y = items[2].split(';') \# with len(y) $>$ 3

items[2] contains at least three ';' $==>$ items[2] is like \verb|[^;]*;[^;]*;[^;]*;.*| 

$==>$ \verb|{0} T {2} [^;]*;[^;]*;[^;]*;.* {3} T {len(items)}|

final regex with items = line.split(','): \verb|^[^,]*,[^,]*,[^;]*;[^;]*;[^;]*;.*$|

with list abstraction: \verb|{0} [^,] {2} ({0} [^;,] {3} T {len}? with delimiter ';') {3} T {len}? with delimiter ','|

\verb|{0}| means \verb|^| (string starts)

\verb|T| means \verb|[^,]*| (everyting except delimiter)

\verb|{0} T {2}| means \verb|^[^,]*,[^,]*|

\verb|{2} [^;]*;[^;]*;[^;]*;.* {3}| results in \verb|^[^,]*,[^,]*,[^;]*;[^;]*;[^;]*;.*|

\verb|{3} T {len(items)}| results in \verb|^[^,]*,[^,]*,[^;]*;[^;]*;[^;]*;.*.*$| (bad because of double .*, maybe check if already .* in end)


\item \textit{x = items[2].split(';')[3]} (version 2)

same as: y = items[2].split(';'); x = y[3]

similar as before: y = items[2].split(';') \# with len(y) $>$ 2

items[2] contains at least 3 ';' $==>$ items[2] is like \verb|{0} [^;]* {4} T {len(items[2])}| with delimiter ';' 

items = line.split(',') \# with len(items) $>$ 2: line is like \verb|{0} [^,]* {3} T {len(items)}| with delimiter ','

with information about items[2]: 

\verb|{0} [^,]* {2} ({0} [^;]* {4} T {len(items[2])} with delimiter ';') {3} T {len(items)}| with delimiter ','

$==>$ \verb|^[^,]*,[^,]*,[^;]*;[^;]*;[^;]*;.*.*$|




\item \textit{x = int(items[2])}

\verb|{0} T {2} \d+ {3} T {len(items)}|

final regex with items = line.split(','): \verb|^[^,]*,[^,]*,\d+.*$|


\item \textit{items = line.split(',') \# with items contains 'chrom'}

final regex: \verb=^chrom(\s*,.*)?|.*,\s*chrom(\s*,.*)?$=

\end{itemize}

\begin{minipage}{\linewidth}
  \centering
\begin{lstlisting}[language=Python2, caption=Example from brca: convert\_tsv\_to\_vcf]
# len(items) > 2 ==> line contains at least two ',' ==> ^[^,]*,[^,]*,.*$
items = line.split(',')
# len(items) > 2
x = items[2]
\end{lstlisting}
\end{minipage}

\begin{minipage}{\linewidth}
  \centering
\begin{lstlisting}[language=Python2, caption=Example from brca: convert\_tsv\_to\_vcf]
# len(items) > 2 ==> line contains at least two ',' ==> ^[^,]*,[^,]*,.*$
items = line.split(',')
# len(items) > 2
x = items[2].split('>')[1]
\end{lstlisting}
\end{minipage}

\section{Code to List Abstraction}

\begin{itemize}
\item problem: \verb|S[i] # with S = {0} T {2} int {3} T {len}?|

How to set the i?

solution: create another list abstraction

\verb|S = {0} T {2} int {3} T {len}? and S = {0} T {i} T {len}?|

wtih i = 1 you get: \verb|S = {0} T {2} int {3} T {len}? and S = {0} T {1} int {2} T {len}?| which can be unified as \verb|S = {0} T {1} int {2} int {3} T {len}?|
\end{itemize}

\end{document}

















