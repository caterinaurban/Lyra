\documentclass[11pt]{article}
\usepackage[a4paper, portrait, margin=1in]{geometry}
\usepackage{fancyhdr}
\usepackage{hyperref}
\usepackage{graphicx}
\usepackage{listings}
\usepackage{mathtools}
\usepackage{color}
\usepackage[dvipsnames]{xcolor}
\usepackage{float}% If comment this, figure moves to Page 2
% Fancy header package for version number
\usepackage[font={small,it}]{caption}

\usepackage{fancyhdr}
%\pagestyle{fancy}
\fancyhf{}
\renewcommand{\headrulewidth}{0pt}
\renewcommand{\footrulewidth}{0pt}
\newcommand{\bigcell}[2]{\begin{tabular}{@{}#1@{}}#2\end{tabular}}
 
\lstdefinelanguage{Python2}{
  language     = Python,
  morekeywords = {assert},
}
 
\lstdefinestyle{mystyle}{
    backgroundcolor=\color{white},   
    commentstyle=\color{ForestGreen},
    keywordstyle=\color{blue},
    numberstyle=\tiny\color{black},
    stringstyle=\color{red},
    basicstyle=\footnotesize,
    breakatwhitespace=false,         
    breaklines=true,                 
    captionpos=b,                    
    keepspaces=true,                 
    numbers=left,                    
    numbersep=5pt,                  
    showspaces=false,                
    showstringspaces=false,
    showtabs=false,                  
    tabsize=2
}

\lstset{style=mystyle, language=Python2}

\begin{document}

\title{List Abstraction}

\maketitle

\pagenumbering{arabic}

\section{Definition}

\begin{itemize}
\item Array Segmentation abstract domain: S(B(E), A, R)
\item Bound abstract domain: B(E)
\item Expression abstract domain: E(X)
\item variable abstract domain: X
\item Array Element abstract domain: A
\item variable environment abstract domain: R
\item convertion array element and variable environment: C(A,R)
\end{itemize}

\section{Definition - Elements}

Question: Should Array element be another S (segmentation) or is only an element of the Assumption Lattice possible?

Answer1: If we only allow Assumption Lattice we have to overapproximate a whole list for list of lists (which can happen if input should be represented as a matrix.

Answer2: If we only allow another S we end up in an infinite loop

Answer3: Array element can either be another S or an element of the Assumption lattice

Question if Answer3 is chosen: How to do the Conversion C of $A \leftrightarrow S$? Is a conversion necessary?

Answer: No Conversion is necessary. The elements in R are the same as in A.

\begin{itemize}
  \item S:
  \begin{itemize}
    \item $b_l$: $(b_1\in B, a_1\in A)$
    \item e: $[(b_1\in B, a_1\in A, \{\_, ?\}), ..., (b_{n-1}\in B, a_{n-1}\in A, \{\_, ?\})]$
    \item $b_l$: $(b_n\in B, \{\_, ?\})$
  \end{itemize}
  \item B: $[b_1, ..., b_n]$ s.t. $b_1 = ... = b_n$ and $b_i \in E$
  \item E: expressions of the form $v + i$
  \item X: $variables \cup \{v_o\} \cup \{*.Length\}$ with * being the array name
  \item A: $a\in AssumptionLattice \vee a\in S$
  \item R: variable $\rightarrow$ $r\in AssumptionLattice \vee r \in S$
  \item C: not needed
\end{itemize}

\section{Example}

\verb|{0} (int, [-inf, inf]) {1 a} (T, [-inf, inf]) {b} (int, [-inf, inf]) {A.Length}|

\begin{itemize}
\item S: 

\verb|{v_0+0} (int, [-inf, inf])|

\verb|{v_0+1 a+0} (T, [-inf,inf]) True|

\verb|{b+0} (int, [-inf, inf]) True|

\verb|{A.Length+0} True|
\item B: 

\verb|[v_0+0]| and \verb|[v_0+1, a+0]| and \verb|[b+0]| and \verb|[A.Length+0}|
\item E: 

\verb|v_0+0| and \verb|v_0+1 a+0| and \verb|b+0| and \verb|A.Length+0|
\item X:

\verb|v_0| and \verb|a| and \verb|b| and  and \verb|A.Length|
\item A: 

\verb|(int, [-inf,inf])| and \verb|(T, [-inf,inf])|
\item R: 

\verb|v_0 -> (int, [0, 0]), a -> (int, [1, 1]),|

\verb|b -> (int, [2, inf]), A.Length -> (int, [3, inf])|
\end{itemize}

\section{Implementation}

\subsection{Initialization}

\begin{itemize}
\item $b_l$: \verb|([v_0+0], (Any, [-inf, inf]))|
\item $e$: \verb|[]|
\item $b_h$: \verb|([A.Length+0], False)|
\end{itemize}

\subsection{Example}

\begin{itemize}
\item $b_l$: \verb|([v_0+0], (int, [-inf, inf]))|
\item $e$: \verb|[([v_0+1, a+0], (T, [-inf, inf]), _), ([b+0], (int, [-inf, inf]), True)]|
\item $b_h$: \verb|([A.Length+0], True)|
\end{itemize}


\end{document}





















