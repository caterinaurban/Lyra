\documentclass[11pt]{article}
\usepackage[a4paper, portrait, margin=1in]{geometry}
\usepackage{fancyhdr}
\usepackage{hyperref}
\usepackage{graphicx}
\usepackage{listings}
\usepackage{mathtools}
\usepackage{color}
\usepackage[dvipsnames]{xcolor}
\usepackage{float}% If comment this, figure moves to Page 2
% Fancy header package for version number
\usepackage[font={small,it}]{caption}

\usepackage{fancyhdr}
%\pagestyle{fancy}
\fancyhf{}
\renewcommand{\headrulewidth}{0pt}
\renewcommand{\footrulewidth}{0pt}
\newcommand{\bigcell}[2]{\begin{tabular}{@{}#1@{}}#2\end{tabular}}
 
\lstdefinelanguage{Python2}{
  language     = Python,
  morekeywords = {assert},
}
 
\lstdefinestyle{mystyle}{
    backgroundcolor=\color{white},   
    commentstyle=\color{ForestGreen},
    keywordstyle=\color{blue},
    numberstyle=\tiny\color{black},
    stringstyle=\color{red},
    basicstyle=\footnotesize,
    breakatwhitespace=false,         
    breaklines=true,                 
    captionpos=b,                    
    keepspaces=true,                 
    numbers=left,                    
    numbersep=5pt,                  
    showspaces=false,                
    showstringspaces=false,
    showtabs=false,                  
    tabsize=2
}

\lstset{style=mystyle, language=Python2}

\begin{document}

\title{Analysis Design Todo}

\maketitle

\pagenumbering{arabic}

\section{Work todo}

\subsection{from examples}

\begin{itemize}
\item reading multiple inputs: a, b = input().split() or values = input().split() (needs implementation of split, multiple inputs and lists)
\item iterating over ranges with non-literal arguments (needs relationships)
\item adding assumptions from loops that execute for sure
        \begin{itemize}
        \item needs store (for assumptions) and relationships in stack
        \item or if we have a loop with "for i in range(10)" we get "not(for i in range)" which is false so we know we can assume bottom here and just take the branch from the loop
        \item problems arize if we have "for i in range(N)", then we would need something like: (assmps, if N $>$ 0)
        \end{itemize}
\item adding conditionals to assumptions: "if a $>$ 10: x = list[n]"
\item Relationships (needs octagon)
\item assumptions about lists (x = a[b] means that $len_a > b$)
\item support other means of input like open(filename), f.readline() etc.
\end{itemize}

\subsection{other}

\begin{itemize}
\item improve design of input checker
\item decide/do evaluation
\item write thesis
\end{itemize}

\section{General input reading example}

\begin{lstlisting}
f = open("filename", "r")
data = []
for line in f:
    values = line.split(",")
    data.append(values)
for datavalues in data:
    val1 = datavalues[1]
    intval1 = int(val1)
\end{lstlisting}

Information to extract:\\

\begin{itemize}
  \item val1 is int (done) (val $\rightarrow$ int)
  \item assumption of val1 applies to datavalues[1] (datavalues $\rightarrow$ $[any, int, any, ..., any]$)
  \item len(datavalues) $\ge$ 1 (len\_datavalues $\rightarrow$ [1, inf])
  \item assumptions of datavalues apply for every element in data (data $\rightarrow$ $[[any, int, any, ..., any], $ $..., [any, int, any, ..., any]]$)
  \item assumption of last element in data applies for values (values $\rightarrow$ $[any, int, any, ..., any]$)
  \item assumptions of values apply for every element in a string splitted by "," (line $\rightarrow$ $[any, int, any, ..., any]$ with delimiter ",")
  \item assumptions of line apply to all elements in f (f $\rightarrow$ [$[any, int, any, ..., any]$ with delimiter ",", ..., $[any, int, any, ..., any]$ with delimiter ","])
  \item f is file with name "filename" (.IN $\rightarrow$ [$[any, int, any, ..., any]$ with delimiter ",", ..., $[any, int, any, ..., any]$ with delimiter ","] with filename = "filename")
\end{itemize}

\begin{enumerate}
  \item before loop: \\
  intval $\rightarrow$ any \\
  val1 $\rightarrow$ any \\
  datavalues $\rightarrow$ \{0\} any \{$len_{datavalues}$\}? \\
  data $\rightarrow$ \{0\} any \{$len_{data}$\}? \\
  
  \item else loop (not(datavalues in data)): \\
  intval $\rightarrow$ any \\
  val1 $\rightarrow$ any \\
  datavalues $\rightarrow$ \{0\} any \{$len_{datavalues}$\}? \\
  data $\rightarrow$ \{0\} BOTTOM \{$len_{data}$\}? \\
  
  \item in loop (1): \\
  intval $\rightarrow$ any \\
  val1 $\rightarrow$ int \\
  datavalues $\rightarrow$ \{0\} any \{$len_{datavalues}$\}? \\
  data $\rightarrow$ \{0\} any \{$len_{data}$\}? \\
  
  \item in loop (2): \\
  intval $\rightarrow$ any \\
  val1 $\rightarrow$ any \\
  datavalues $\rightarrow$ \{0\} any \{1\} int \{2\} any \{$len_{datavalues}$\}? \\
  data $\rightarrow$ \{0\} any \{$len_{data}$\}? \\
  
  \item loop header (datavalues in data): \\
  intval $\rightarrow$ any \\
  val1 $\rightarrow$ any \\
  datavalues $\rightarrow$ \{0\} any \{$len_{datavalues}$\}? \\
  data $\rightarrow$ \{0\} (\{0\} any \{1\} int \{2\} any \{$len_{datavalues}$\}?) \{$len_{data}$\} \\
  
  \item merged: \\
  intval $\rightarrow$ any \\
  val1 $\rightarrow$ any \\
  datavalues $\rightarrow$ \{0\} any \{$len_{datavalues}$\} \\
  data $\rightarrow$ \{0\} (\{0\} any \{1\} int \{2\} any \{$len_{datavalues}$\}?) \{$len_{data}$\}? \\
  
  \item before loop: \\
  data $\rightarrow$ \{0\} (\{0\} any \{1\} int \{2\} any \{$len_{datavalues}$\}?) \{$len_{data}$\}? \\
  values $\rightarrow$ any
  line $\rightarrow$ any
  f $\rightarrow$ \{0\} any \{$len_{f}$\}? \\
  
  \item else loop (not line in f): \\
  data $\rightarrow$ \{0\} (\{0\} any \{1\} int \{2\} any \{$len_{datavalues}$\}?) \{$len_{data}$\}? \\
  values $\rightarrow$ any \\
  line $\rightarrow$ any \\
  f $\rightarrow$ \{0\} BOTTOM \{$len_{f}$\}? \\
  
  \item in loop (1): \\
  data $\rightarrow$ \{0\} (\{0\} any \{1\} int \{2\} any \{$len_{datavalues}$\}?) \{$len_{data}$\}? \\
  values $\rightarrow$ \{0\} any \{1\} int \{2\} any \{$len_{datavalues}$\}? \\
  line $\rightarrow$ any \\
  f $\rightarrow$ \{0\} any \{$len_{f}$\}? \\
  
  \item in loop (2): \\
  data $\rightarrow$ \{0\} (\{0\} any \{1\} int \{2\} any \{$len_{datavalues}$\}?) \{$len_{data}$\}? \\
  values $\rightarrow$ any \\
  line $\rightarrow$ \{0\} any \{1\} int \{2\} any \{$len_{datavalues}$\}? with delimiter "," \\
  f $\rightarrow$ \{0\} any \{$len_{f}$\}? \\
  
  \item loop header (line in f): \\
  data $\rightarrow$ \{0\} (\{0\} any \{1\} int \{2\} any \{$len_{datavalues}$\}?) \{$len_{data}$\}? \\
  values $\rightarrow$ any \\
  line $\rightarrow$ any \\
  f $\rightarrow$ \{0\} \{0\} any \{1\} int \{2\} any \{$len_{datavalues}$\}? with delimiter "," \{$len_{f}$\} \\
  
  \item merged: \\
  data $\rightarrow$ \{0\} (\{0\} any \{1\} int \{2\} any \{$len_{datavalues}$\}?) \{$len_{data}$\}? \\
  values $\rightarrow$ any \\
  line $\rightarrow$ any \\
  f $\rightarrow$ \{0\} (\{0\} any \{1\} int \{2\} any \{$len_{datavalues}$\}? with delimiter ",") \{$len_{f}$\}? \\
  
  \item open file: \\
  data $\rightarrow$ \{0\} (\{0\} any \{1\} int \{2\} any \{$len_{datavalues}$\}?) \{$len_{data}$\}? \\
  values $\rightarrow$ any \\
  line $\rightarrow$ any \\
  f $\rightarrow$ any \\
  .IN $\rightarrow$ FORALL: (\{0\} any \{1\} int \{2\} any \{$len_{datavalues}$\}? with delimiter ",") with filename "filename" \\
  
\end{enumerate}

\section{Relationship example}

\begin{lstlisting}
f = open("filename", "r")
num = int(f.readline())
for line in range(num):
    x = float(f.readline())
    print(x)
\end{lstlisting}

Problem: how to realize that f is input? \\
Idea: annotate each input where it came from (input(), f, ...) if we realize that for example f is not a call to open() (or some other means of input) we could remove those entries

Problem: need some relational domain for example if for i in range(N) the N might need to be substituted with N = N + 1 etc.\\
Idea: if needed have a store entry $iter_1 \rightarrow N$ etc so that we can substitute those iteration variable at the end (with for example an input).\\


\begin{enumerate}
  \item before loop: \\
  x $\rightarrow$ any \\
  line $\rightarrow$ any \\
  num $\rightarrow$ any \\
  f $\rightarrow$ any \\
  .IN $\rightarrow$ []
  
  \item else loop (not(line in range(num))): \\
  x $\rightarrow$ any \\
  line $\rightarrow$ any \\
  num $\rightarrow$ int, [-inf, 0] \\
  f $\rightarrow$ any \\
  .IN $\rightarrow$ []
  
  \item in loop: \\
  x $\rightarrow$ any \\
  line $\rightarrow$ any \\
  num $\rightarrow$ int, [1, inf] \\
  .IN $\rightarrow$ [float]
  
  \item merged \\
  x $\rightarrow$ any \\
  line $\rightarrow$ any \\
  num $\rightarrow$ int \\
  .IN $\rightarrow$ [num x float]
  
  \item after loop (1) \\
  x $\rightarrow$ any \\
  line $\rightarrow$ any \\
  num $\rightarrow$ int \\
  .IN $\rightarrow$ [int, .IN[-2] x float]
\end{enumerate}

\end{document}





















