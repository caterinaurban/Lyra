\documentclass[11pt]{article}
\usepackage[a4paper, portrait, margin=1in]{geometry}
\usepackage{fancyhdr}
\usepackage{hyperref}
\usepackage{graphicx}
\usepackage{listings}
\usepackage{mathtools}
\usepackage{color}
\usepackage[dvipsnames]{xcolor}
\usepackage{float}% If comment this, figure moves to Page 2
% Fancy header package for version number
\usepackage[font={small,it}]{caption}

\usepackage{fancyhdr}
%\pagestyle{fancy}
\fancyhf{}
\renewcommand{\headrulewidth}{0pt}
\renewcommand{\footrulewidth}{0pt}
\newcommand{\bigcell}[2]{\begin{tabular}{@{}#1@{}}#2\end{tabular}}
 
\lstdefinelanguage{Python2}{
  language     = Python,
  morekeywords = {assert},
}
 
\lstdefinestyle{mystyle}{
    backgroundcolor=\color{white},   
    commentstyle=\color{ForestGreen},
    keywordstyle=\color{blue},
    numberstyle=\tiny\color{black},
    stringstyle=\color{red},
    basicstyle=\footnotesize,
    breakatwhitespace=false,         
    breaklines=true,                 
    captionpos=b,                    
    keepspaces=true,                 
    numbers=left,                    
    numbersep=5pt,                  
    showspaces=false,                
    showstringspaces=false,
    showtabs=false,                  
    tabsize=2
}

\lstset{style=mystyle, language=Python2}

\begin{document}

\title{Input Assumptions}

\maketitle

\pagenumbering{arabic}

\section{Desgin}

Problem: join with different number of inputs:

\begin{lstlisting}[numbers=none]

a = int(input())
if a > 10:
    # INPUT = [float, int]
    b = int(input())
else:
    # INPUT = [float]
    b = a + 1
# INPUT = [float]
c = float(input())
# INPUT = []
\end{lstlisting}

correct solution: \verb|if a > 10: [float, int, int], else: [float, int]|

current solution: if we need to join two input-lists with different length we set the Input Assumption to bottom. If we encounter another input() we create a new InputAssumption list and add the current assumption, this way we keep at least information about the first couple of assumptions.

\end{document}






















