\documentclass[11pt]{article}
\usepackage[a4paper, portrait, margin=1in]{geometry}
\usepackage{fancyhdr}
\usepackage{hyperref}
\usepackage{graphicx}
\usepackage{listings}
\usepackage{mathtools}
\usepackage{color}
\usepackage[dvipsnames]{xcolor}
\usepackage{float}% If comment this, figure moves to Page 2
% Fancy header package for version number
\usepackage[font={small,it}]{caption}

\usepackage{fancyhdr}
%\pagestyle{fancy}
\fancyhf{}
\renewcommand{\headrulewidth}{0pt}
\renewcommand{\footrulewidth}{0pt}
\newcommand{\bigcell}[2]{\begin{tabular}{@{}#1@{}}#2\end{tabular}}
 
\lstdefinelanguage{Python2}{
  language     = Python,
  morekeywords = {assert},
}
 
\lstdefinestyle{mystyle}{
    backgroundcolor=\color{white},   
    commentstyle=\color{ForestGreen},
    keywordstyle=\color{blue},
    numberstyle=\tiny\color{black},
    stringstyle=\color{red},
    basicstyle=\footnotesize,
    breakatwhitespace=false,         
    breaklines=true,                 
    captionpos=b,                    
    keepspaces=true,                 
    numbers=left,                    
    numbersep=5pt,                  
    showspaces=false,                
    showstringspaces=false,
    showtabs=false,                  
    tabsize=2
}

\lstset{style=mystyle, language=Python2}
\lstset{
    escapeinside={@}{@},          % if you want to add LaTeX within your code
}

\begin{document}

\title{Input Collection Design}

\maketitle

\pagenumbering{arabic}

\section{Issues}

\subsection{Issue 1}

\begin{itemize}
\item pop if else: \verb/[a]|[b]: [b, a]/
\item pop loop: \verb/[a]|[b]: [4x[b], a]/
\item problem join: how to distinguish if: loop; else: loop; and loop;loop;

if: loop; else: loop; \verb/join [4x[b]] and [4x[b]]/

loop;loop; \verb/join [4x[b]] and [4x[b]]/

\item idea: use a placeholder for every loop with possible input (so that loops further up have information that there is a loop further down)
\item disadvantage: if there is no input end up with placeholders in the end, need to clean up
\end{itemize}

\subsection{Issue 2}

\begin{itemize}
\item problem join: how to distinguish second/third loop iteration if new info from further down and how to distinguish third iteration join with if/else join with information loss

info first iteration else: \verb/[a]/

first join: \verb/BOTTOM and [a] is [a]/

2nd iteration: \verb/[a]|[b] gets [b, a]/

2nd join, new info (without placeholder): \verb/[b, a] and [c, a] (should get [b, c, a])/

2nd join, new info (with placeholder): \verb/[b, NONE, a] and [c, a] (should get [b, c, a])/

3rd iteration: \verb/[b, a]|[b] gets [b, b, a]/

3rd join, new info: \verb/[b, b, c, a] and [b, c, a] (should get [b, c, a])/

\item idea: first iteration produces placeholder, second iteration we substiture placeholder, third iteration nothing is done
\item problem: how to do this? placeholder is produced when doing pop() and we cannot distinguish the loop iteration. if we try to do this with the join, how can we distinguish a join of loops where some have no input and a join of loop iterations where we have multiple placeholders
\end{itemize}

\subsection{Issue 3}

\begin{itemize}
\item problem pop: how to know if placeholder from before is other loop or same loop ?

loop1, first iteration else: \verb/[a]|[]/

loop1, first pop: \verb/[LOOP, a]/

loop1, first join: \verb/BOTTOM and [LOOP, a] is [LOOP, a]/

loop2, first iteration else: \verb/[LOOP, a]|[]/

loop2, first pop: \verb/[LOOP, LOOP, a]/

loop2, first join: \verb/BOTTOM and [LOOP, LOOP, a] is [LOOP, LOOP, a]/

loop1, second iteration with input: \verb/[LOOP, a]|[b]/

loop1, second pop with input: \verb/[b, a]|[b]/

loop1, second iteration without input: \verb/[LOOP, a]|[]/

loop1, second pop without input: \verb/[LOOP, a]|[] NOT: [LOOP, LOOP, a]|[]/

\end{itemize}

\subsection{Issue 3}
\begin{itemize}
\item how to distinguish if: loop; else: loop with input; from joining loop with else?

if: loop; else: loop with input; \verb/join [LOOP] and [4x[b]] should be infoloss/

loop and loop else: \verb/join [LOOP] and [4x[b]] shoudl be [4x[b]/

\item idea: remove LOOP if joined with LOOP again

\item problem: loop with input; loop without input; because in first round we get \verb/[None] and [None, None]/ and in second round the first loop gets \verb/[]/ so in the second loop we have to join \verb/[None, None]|[b] -> [b, None] and []|[] -> [None]/

\item idea: pop(1) if lengths are different ... not true if joining if and else of loops

\item idea: add flag to input assumption if it should be joined like an if/else or loop (when doing pop), therefore you know in what case you are when doing the join. Also always only join the top element of the stack

\end{itemize}

\subsection{Issue 4}
\begin{itemize}
\item if i pop \verb/[LOOP]|10x[LOOP]/ i end up with \verb/[10x[LOOP]]/ but now i get a problem if i need to pop \verb/[10x[LOOP]]|[2x[b]]/ because it should go inside the outer loop. Just replacing the loop with the inner does not work because it might be, that the inner loop is just from a 3rd iteration or so. Other idea is not to pop in the case of \verb/[LOOP]|10x[LOOP]/ und just wait until inner loop is done first. But if the inner loop has no input we will just end up with the placeholder every time. Removing the placeholder when we pop \verb/[LOOP]|[LOOP]/ would also not work because it might be, that...

\end{itemize}

\end{document}






















